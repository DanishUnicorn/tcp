\setcounter{chapter}{0}
\setcounter{section}{0}
\chapter{Lecture Notes}
\setlength{\headheight}{12.71342pt}
\addtolength{\topmargin}{-0.71342pt}

\section{Lecture 01 - 02/09-2025}
\subsection{The Tropical Environment} 

\subsubsection{Aim} 
\begin{itemize} 
    \item Overview the most important aspects of tropical climates. 
    \item Ability to figure out how the climate is likely to be in certain places in the tropics. 
    \item Idea of which crop you can grow. 
\end{itemize}

\subsection{What Determines the Climate?} 
The climate is determined by several factors, including temperature and precipitation. Key aspects are the yearly average temperature and the yearly range in temperature, as some areas experience a larger difference between the highest and lowest temperatures than others. Similarly, average precipitation is important, but the yearly variation in rainfall also plays a significant role.
\subsubsection*{Core takeaway:} 
Climate is primarily defined by temperature and precipitation, considering both yearly averages and seasonal variations. Likely exam-relevant.

\subsection{Classification: Latitudes} 

\begin{itemize} 
    \item Tropical zone from 0\textdegree–23.5\textdegree (between the tropics) latitude: Here, solar radiation reaches the ground nearly vertically, more water evaporates, and the air is often moist. A dense cloud cover reduces the effect of solar radiation on ground temperature. 
    \item Subtropics from 23.5\textdegree–40\textdegree latitude: These regions receive the highest radiation in summer, have relatively thin cloud cover, and receive less moisture. 
    \item Temperate zone from 40\textdegree–60\textdegree latitude: This zone is characterized by significantly differing seasons and day lengths, less frequent climate extremes, a more regular distribution of precipitation, and a longer vegetation period. 
    \item Cold zone from 60\textdegree–90\textdegree latitude: The poles in this zone receive less heat through solar radiation, and day length varies the most. Vegetation is only possible during a few months and is often sparse. 
\end{itemize}

\subsubsection*{Core takeaway:} 
Earth's climate zones are classified by latitude, each with distinct characteristics regarding solar radiation, temperature, precipitation, and vegetation periods. Likely exam-relevant.


\subsection{Circles of Latitude and Longitude} 
\subsubsection{Earth's Movement and Tropical Rain Belt} 
The Earth spins around its axis, akin to a top, a process known as Earth's rotation. Simultaneously, it orbits or revolves around the Sun. The tropical rain belt runs along the equator and extends to about the Tropic of Cancer (23.5\textdegree north latitude) and Tropic of Capricorn (23.5\textdegree south latitude). By approximately 30\textdegree north and south latitude, the air cools enough to sink back to the surface, creating high pressure (H) and drier conditions.
\subsubsection{Earth's Orbit and Solar Energy} The Earth's revolution around the sun takes 365.24 days. At the equator, the Earth rotates at roughly 1,700 km per hour. The Earth is closest to the sun (perihelion) on January 3rd at 147 million km, moving faster at 27 km/s. It is furthest from the sun (aphelion) on July 4th at 152 million km, moving slower. Solar energy is relatively constant, approximately 400 W/m$^2$/year. About 300 W/m$^2$/year is lost as terrestrial re-radiation, leaving a surplus of 100 W/m$^2$ at the surface. Most of the radiation is absorbed by the Earth and warms it. Some of the outgoing infrared radiation is trapped by the Earth’s atmosphere, which also contributes to warming.

\subsubsection*{Core takeaway: }
Earth's rotation and revolution influence climate patterns, including the tropical rain belt, and its interaction with solar energy dictates global temperatures. Likely exam-relevant. 


\subsection{The Tropics} 
The tropics are characterized by a high input of solar radiation and high maximum temperatures, with little variation in temperature. Water supply is the most significant variable, marked by high rainfall variability and high rainfall intensity. The tropics cover 42\% of the Earth's surface. 
\subsubsection{Characterize the tropics !} 
\subsubsection{Precipitation} 
Precipitation patterns in the tropics include: 
\begin{itemize} 
    \item Wet climate (between 5\textdegree and 10\textdegree of the equator). 
    \item Wet dry climate (between 10\textdegree and 20\textdegree). 
    \item Two wet seasons: typically 1000-2000 mm (e.g., Salvador, Abidjan). 
    \item Two shorter rainy seasons (e.g., Nairobi). 
    \item One long rainy season: monsoonal, 750-1500 mm (e.g., Manila). 
    \item One short rain season: 250-750 mm (e.g., Darwin, Hyderabad). 
    \item Dry climate (e.g., Alice Springs, Lima, Khartoum)
\end{itemize}


\subsubsection*{Core takeaway:} 
The tropics receive high solar radiation and experience consistent high temperatures, with water supply and significant rainfall variability being defining features across different precipitation zones. Likely exam-relevant.


\subsection{Three Major Biomes} 
A biome is defined as a community of similar plants and animals occupying a large area. The three major biomes are Forest, Savanna, and Desert. 

\subsubsection{Tropical biomes and annual precipitation (mm)} Tropical biomes exhibit extremely high biodiversity, encompassing 50\% of the world’s terrestrial plant and animal species, despite covering only about 6\% of the world’s land area.

\subsubsection*{Core takeaway:} 
The tropics host three major biomes—Forest, Savanna, and Desert—which are critical for global biodiversity, harboring half of the world's terrestrial species in a small land area. Likely exam-relevant.


\subsection{Deforestation} 
Before human intervention, rainforests covered 15\% of the Earth's land area, but today they cover only 6\%. In the last 200 years, the total area of rainforest has decreased from 1,500 million hectares to less than 800 million hectares. A third of tropical rainforests have been destroyed in just the last 50 years. Approximately 119,000 - 150,219 km$^2$ are lost each year, affecting the world's most spectacular ecosystems.

\subsubsection*{Core takeaway: }
Deforestation has drastically reduced tropical rainforest coverage, leading to a significant loss of these vital ecosystems globally. Likely exam-relevant.


\subsection{Daily Weather Cycle in the Tropical Rainforest} 
In the morning, the sun shines and heats up the ground, causing hot and wet air to rise. In the afternoon, dark clouds form, bringing rain and thunderstorms to the rainforest.


\subsection{Prevailing Winds} 
\subsubsection{Latitudinal Variation in Evapotranspiration and Precipitation} 
(figure, see slide 9)
\subsection{Remember!} 
\begin{itemize} 
    \item Hot air weighs less than cold air. 
    \item Hot air can contain more water than cold air. 
    \item Air will flow from areas of high pressure towards areas with low pressure. 
    \item Condensation of water releases energy. 
    \item The temperature of the air drops approximately 1 degree for every 100 m, or 0.5 degrees if the air contains water. 
    \item Objects moving in the northerly or southerly direction will be deflected clockwise in the northern hemisphere and counter-clockwise in the southern hemisphere (Coriolis force) (see also Slide 10). 
\end{itemize}

\subsubsection*{Core takeaway:} 
Atmospheric dynamics, driven by temperature, pressure, and the Coriolis force, dictate air movement, moisture content, and temperature changes critical for understanding weather patterns. Likely exam-relevant.


\subsection{Coriolis Force} 
When the Earth rotates, a point close to the equator moves much faster than a point at one of the poles. This movement creates specific patterns on Earth and affects winds and ocean currents.

\subsubsection*{Core takeaway: }
The Coriolis force, a result of Earth's rotation, deflects moving objects and significantly influences global wind and ocean current patterns. Likely exam-relevant.


\subsection{Tropical Storms} 
Tropical storms include Hurricanes (in the Caribbean and United States) and Typhoons (in the Pacific Ocean). These storms are characterized by wind speeds exceeding 115 km/hour, low pressure, and a circular pattern of isobars with a diameter of 150-650 km. They bring extreme rainfall (up to 200 mm/day) and steep gradients that produce high wind speeds. 

\subsubsection{Cyclones Around Australia}
\subsection{Monsoons} 
Monsoons are large-scale sea breezes that occur when the temperature on land is significantly warmer or cooler than the temperature of the ocean. These temperature imbalances happen because oceans and land absorb heat in different ways.

\subsubsection*{Core takeaway:} 
Tropical storms like hurricanes and typhoons are intense low-pressure systems with high winds and extreme rainfall, while monsoons are seasonal wind shifts caused by differential heating of land and sea. Likely exam-relevant.


\subsection{Southeast Asian Rainforests} 
Southeast Asian rainforests experience four different seasons: the winter northeast monsoon, the summer southwest monsoon, and two inter-monsoon seasons. 

\begin{itemize} 
    \item The northeast monsoon season (November to March) has steady winds from the north or northeast, originating from Siberia, which bring typhoons and other severe weather. The east coasts of the Southeast Asian islands receive heavy rains during this time. 
    \item The southwest monsoon season (May to September) has less wind and is slightly drier, though it still rains every day. 
    \item During the inter-monsoon seasons, the winds are light. All seasons are hot and humid, with very little seasonal variation in temperature. 
\end{itemize}

\subsubsection*{Core takeaway:} 
Southeast Asian rainforests experience distinct monsoon seasons driven by regional wind patterns, resulting in varied rainfall but consistently hot and humid conditions year-round. Likely exam-relevant.


\subsection{Tropical Rainforests} 
Tropical rainforests are characterized by a type of tropical climate with no dry season, meaning all months have an average precipitation value of at least 60 mm (2.4 in). There are no distinct summer or winter seasons; it is typically hot and wet throughout the year, with both heavy and frequent rainfall. Around the equator, there are two seasons with heavy rainfall, receiving up to 10 meters a year. As one moves away from the equator, it becomes a bit drier in some months, but there is still more than 2 meters of rain annually. Most of the rainfall does not reach the ground directly, as the trees act as a canopy and catch the rain. 

\subsubsection{Rainforest Burned Down in South America} (image, see slide 14)
\subsubsection*{Core takeaway:} 
Tropical rainforests are defined by continuous high rainfall, consistent high temperatures year-round, and the significant role of their dense canopy in intercepting precipitation. Likely exam-relevant.


\subsection{Tropical Desert} Major tropical desert areas include the Sahara and Kalahari deserts in Africa, Arabian, Iranian and Thar Deserts in Asia, Arizona and Mexican deserts in North America, and the Great Australian Desert. 

\subsubsection{Oasis with Date Palm} (image, see slide 15) 

\subsubsection{External Resources / Ecosystem Map} 
\textit{[Requires further research: This section primarily provides links to external resources (YouTube and a NOAA ecosystem map) and does not contain descriptive content within the slides themselves.]}

\subsection{A Simple Illustration of the Major Crop Types in Relation to Climate} 
\textit{[Requires further research: This slide title suggests an illustration but the content is not provided.]}

\subsubsection*{Core takeaway:} 
Tropical deserts are extensive arid regions found across multiple continents, characterized by very low precipitation and extreme temperatures. Likely exam-relevant.