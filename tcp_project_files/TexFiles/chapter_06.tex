\chapter{Exam Questions and Answers}
\setlength{\headheight}{12.71342pt}
\addtolength{\topmargin}{-0.71342pt}

This chapter of the course notes compiles the exam questions for the course held in November 2025, along with their respective answers prepared by me. The purpose of this section is twofold: firstly, to provide a reflective exercise that consolidates understanding of the course material; and secondly, to document my comprehension of the course topics as assessed through the exam questions.

\vspace{1em}
To ensure citation accuracy and academic transparency, NotebookLM has been employed as the primary generative AI platform. Its use has focused on verifying that all citations accurately reference the uploaded course materials and lecture slides provided by the professors. Beyond citation control, this section also represents an ongoing exploration of prompt engineering - refining interaction design to optimise AI output quality, precision, and academic reliability. Through this approach, the work aims to maintain a high academic standard while enhancing clarity, structure, and depth in written responses.

\vspace{1em}
There are a total of 17 questions in the exam, each comprising between three and five sub-questions. The numbering of the sections in this chapter corresponds directly to the numbering of the exam questions, ensuring a clear and consistent structure throughout. Questions 1-9 address aspects related to crop physiology, while questions 10-17 focus on fruit quality, maturity, and usability. Each question is presented below, followed by its respective sub-questions and answers.

\newpage
\section{Question 1 - Highland Crops}
\textbf{How would you make the association between crop calendar and climate change. Specify links between crop phenology and climate knowledge. You can use an example:}

\vspace{1em}
Crop phenology describes the timing of developmental stages (e.g.germination, flowering, and maturity) mainly controlled by temperature and moisture. These patterns form the scientific basis of traditional crop calendars, which guide when farmers sow and harvest to match favourable climatic windows. Phenological events follow the principle of thermal time, meaning the accumulation of degree days above a base temperature determines progression between stages.

\vspace{1em}
For example, cañahua germinates at about -0.9 \textdegree C and requires roughly 476 \textdegree C hours, demonstrating its adaptation to cold, highland environments. In the Bolivian Altiplano, short rainy seasons and low night temperatures restrict the growing period; farmers therefore sow between September and November so crops can establish before drought sets in.

\vspace{1em}
Climate change disrupts this synchrony. Rising mean temperatures and altered rainfall patterns shift the timing of phenological events-flowering, grain filling, and maturity often occur earlier, shortening growth duration and reducing yield potential. Observations in ahipa show that delayed sowing shortens the vegetative period and lowers yield, while long-term comparisons between the 1990s and 2010s reveal shorter crop cycles across many Andean species.

\vspace{1em}
Farmers increasingly adapt their calendars and crop choices: replacing long-cycle species with shorter, marketable varieties; advancing sowing dates; or relying on predictive models based on degree-day accumulation and rainfall forecasts. These models combine phenological and climate data to simulate optimal sowing windows under future scenarios, turning traditional knowledge into dynamic climate-adaptation tools.

\vspace{1em}
Genetic diversity among native highland crops, such as frost-tolerant Solanum species or cold-adapted quinoa relatives, enhances resilience by offering flexible phenological responses to temperature shifts. As warming raises the frost line and modifies the thermal gradient with altitude, such traits become critical for maintaining yield stability.

\vspace{1em}
In summary, the crop calendar embodies an integration of phenological understanding and climate knowledge. It translates thermal thresholds, rainfall patterns, and local experience into adaptive management decisions. Under accelerating climate change, linking phenology to predictive climate data allows highland farmers to safeguard productivity and sustain traditional systems through informed, flexible timing of their agricultural cycles.

\newpage
\section{Question 2 - Highland Crops}
\textbf{Which are the social constrains when introducing new varieties? And other factors of agronomic importance? }

\vspace{1em}
Introducing new varieties in highland systems faces both social and agronomic constraints. Socially, farmer decisions are shaped by culture, markets, and knowledge systems. Many native Andean species such as ahipa or mauka are culturally undervalued and historically labelled as “poor man’s food,” reducing consumer demand. Urbanisation and dietary modernisation have weakened traditional consumption patterns, accelerating preference toward processed staples and imported crops.

\vspace{1em}
Market access is another barrier. Limited marketing channels, poor transport infrastructure, and unstable prices discourage adoption of crops without reliable buyers. Rural families therefore prioritise short-cycle, high-value crops that provide frequent cash flow, while long-cycle or niche species are viewed as economically risky. Farmer risk aversion is a key factor; when climate and income uncertainties are high, producers rarely gamble on unfamiliar varieties. Loss of traditional knowledge due to youth migration further undermines skills in cultivation, seed handling, and processing of indigenous crops. Weak extension networks and insufficient institutional support also limit dissemination and acceptance of new or improved varieties.

\vspace{1em}
Agronomically, highland crops show exceptional tolerance to frost, drought, and poor soils - for example, cañahua and native Solanum potatoes perform at high altitude with minimal external inputs. Yet several biological bottlenecks hinder broader adoption: seed shattering in cañahua, virus susceptibility in tubers like ulluco and mashua, and the trade-off between vegetative and reproductive growth in ahipa. Seed quality and availability remain major obstacles, as formal seed systems rarely include these species. Improved management practices such as: e.g. pruning, soil fertility enhancement, or rhizobia inoculation, can significantly increase yields, but research investment and extension services are limited relative to commercial staples.

\vspace{1em}
In summary, adoption of new varieties in the highlands depends not only on biological adaptation but also on cultural acceptance, market incentives, and institutional support. Successful introduction requires addressing consumer perception, securing stable value chains, strengthening seed systems, and providing technical guidance to farmers - linking agronomic improvement with social relevance and economic viability.


\newpage
\section{Question 3 - Intercropping}
\textbf{Explain the concept of intercropping and provide potential benefits of practicing intercropping in agriculture.}

\vspace{1em}
Intercropping is the practice of growing two or more crop species simultaneously on the same land, often arranged in rows, strips, or mixed patterns. In traditional Andean systems, it remains a cornerstone of sustainable production, integrating food, fodder, and cash crops in one field. Typical combinations include ahipa with maize or onion, mauka with maize and beans, and mixed tuber systems of oca, ulluco, and mashua.

\vspace{1em}
The agronomic rationale is based on complementary resource use. Species differ in rooting depth, canopy structure, and nutrient demand, allowing more efficient capture of light, water, and soil nutrients. This spatial and temporal complementarity often raises the land equivalent ratio above 1, meaning greater overall productivity than sole crops on the same area. Legume-non-legume pairings, such as ahipa-maize, also improve soil fertility through biological nitrogen fixation, reducing fertilizer needs.

\vspace{1em}
Ecologically, intercropping enhances system stability. Mixed canopies disrupt pest and disease cycles, dilute host density, and provide microclimatic buffering against frost and radiation stress. Mauka intercropped with maize, for example, benefits from root antimicrobial compounds that suppress soil pathogens, while the maize canopy offers shade protection. Such interactions increase yield resilience under the variable highland climate.

\vspace{1em}
Beyond agronomic gains, intercropping sustains agrobiodiversity and traditional polyculture knowledge. It diversifies harvests, spreads economic risk, and maintains soil cover that prevents erosion on steep slopes. These features embody ecological intensification-higher productivity with lower external inputs-making intercropping a key strategy for both food security and environmental sustainability in mountain agriculture.

%1. Explaining the Concept of Intercropping
%The sources describe intercropping as the practice of growing multiple crops together within the same plot or farming system. This is associated with traditional farming systems or polyculture systems used by smallholder farmers in areas like the Inter-Andean Valleys (IAV) and the Bolivian Altiplano.
%Key examples from the sources illustrating the concept include:
%• Ahipa and Other Crops: Farmers cultivate ahipa (Pachyrhizus ahipa) in traditional polyculture systems. These plots may include tomato (Solanum lycopersicum L.), onion (Allium cepa L.), fennel (Foeniculum vulgare Mill.), arracacha, groundnut (Arachis hypogaea L.), cassava (Manihot esculenta Crantz), white giant maize corn (Zea mays L. var. cuzcoensis Kornicke), and achoccha (Cyclanthera pedata Schrad.).
%• Mauka and Maize: Mauka (Mirabilis expansa) is frequently intercropped with maize. It is also found cultivated in assortment with maize, beans (Phaseolus vulgaris L.), yacón, and squash (Cucurbita pepo L.). Sometimes, mauka grows spontaneously in wheat crops.
%• Arracacha in Mixed Systems: Arracacha is either monocropped or in association with maize, or beans (Phaseolus spp.), or intercropped with other crops in the region. In the community of Lloja (Bolivia), arracacha plants were observed intercropped with cassava.
%• Cassava in Polyculture: Cassava is typically cultivated in a polyculture system with other floodplain crops.
%• General Andean Practice: The tuber-bearing plants oca, ulluco, and mashua are traditionally planted together either mixed or in adjacent plots.
%• Achira in Familiar Cropping Systems: Smallholders practice intercropping and polyculture involving achira (Canna indica L.).
%2. Potential Benefits of Practicing Intercropping
%The sources associate intercropping and mixed cropping with several significant benefits, primarily related to agricultural sustainability, resource use efficiency, and risk reduction.
%A. Sustainable Land Use and Soil Health
%• Biological Nitrogen Fixation: Intercropping legumes, specifically the Pachyrhizus species, with other crops promotes soil fertility. As an N2 -fixing legume, ahipa does not require nitrogen fertilization. The genus Pachyrhizus has an efficient symbiosis with nitrogen-fixing Rhizobium and Bradyrhizobium bacteria, providing plants with nitrogen. This feature allows the crop to form an integral part of a sustainable land-use system, from both an ecological and a socioeconomic standpoint.
%• Crop Rotation: Ahipa is cited as an interesting crop to include in rotations. Farmers practice crop rotation, often shifting from a root or tuber crop with nematode problems to a grain crop, e.g., maize, the next year.
%B. Crop Protection and Resilience
%• Pest and Disease Management: The sources emphasize the inherent resilience to pests and diseases in these mixed systems. Mauka's aptitude for self-defense is noted, with some informants suggesting that intercropping maize may shelter mauka from frost damage. The fact that mauka is generally healthy and the ribosome-inactivating proteins synthesized from its storage roots have an antimicrobial effect against root-rotting microorganisms reinforces its suitability for mixed systems.
%• Yield Reliability: The combination of a tuberous root crop (like Pachyrhizus) with a legume offers the yield reliability of a tuberous root crop, combined with the high sustainability of a legume.
%C. Optimized Resource Use
%• Multiple Outputs (Fodder/Food): Intercropping systems allow farmers to maximize outputs from small plots of land. Achira leaves can be a good alternative for fodder in rural households. The dried hay of Pachyrhizus is used as fodder. The ability of mauka to be used for both leaves and roots also enhances resource utility in these mixed systems.
%• Adaptability: The cultivated Pachyrhizus groups easily adapt to small-farmers systems, as it can be intercropped with maize and bean. This versatility is crucial in the challenging Andean environment.
%D. Preservation of Agrobiodiversity
%• Maintaining Diversity: Intercropping is a foundational element of the traditional farming systems (e.g., chiru, ananta, and k'ata), which enable farmers to maintain a high agro-biodiversity. High agro-biodiversity cultivated on small land sizes is characteristic of these communities. The practice of growing many species together contributes to the conservation of the environment and biodiversity.

\newpage
\section{Question 4 - Intercropping}
\textbf{Discuss the challenges and potential disadvantages of intercropping in modern agricultural systems. Provide examples of situations where intercropping may not be the best strategy.}

\vspace{1em}
Although intercropping provides ecological and sustainability benefits, it faces important challenges in modern agricultural systems. Agronomically, competition between component species can reduce efficiency when growth rates or canopy structures differ. Fast-growing crops like mauka may overshadow companions, and in systems targeting multiple products-such as achira for both leaves and rhizomes-energy partitioning often compromises the main yield. High planting densities in mixed plots can also limit root development, as observed in ahipa, lowering harvest index compared to monocultures.

\vspace{1em}
Management complexity is another barrier. Intercropping requires detailed spatial design, staggered sowing, and differentiated harvesting schedules, which are difficult to standardise. Labour demand is typically higher, particularly for tasks like selective pruning or manual weeding. These operations increase production costs, making intercropping less attractive in large-scale systems where mechanisation and uniformity are essential for profitability.

\vspace{1em}
Economic and industrial constraints further discourage adoption. Mixed harvests generate non-uniform raw materials that are incompatible with automated processing and value chains demanding bulk, standardized inputs. For example, industrial-scale starch extraction from ahipa or achira favours monocultures to ensure consistency and efficiency. Similarly, mechanised cereal or vegetable systems cannot easily integrate intercropping due to machinery calibration, planting geometry, and harvesting logistics.

\vspace{1em}
In such contexts-where mechanisation, uniformity, and rapid economic returns dominate-intercropping is rarely the best strategy. It is better suited to smallholder or ecological systems prioritising resilience, soil fertility, and risk reduction rather than maximum yield. Competitive or space-demanding crops like mauka may instead be planted as border strips or rotational species to complement rather than compete with main crops.

\vspace{1em}
In summary, while intercropping enhances biodiversity and ecological sustainability, its practical limitations-competition, labour intensity, management complexity, and incompatibility with modern mechanised chains-restrict its use in high-input agriculture. The challenge lies in adapting its ecological principles into scalable forms of diversified cropping systems that balance productivity with sustainability.


%Challenges and Potential Disadvantages of Intercropping
%The sources highlight several critical constraints related to agronomy, labor, and economic pressure that challenge the sustainability and adoption of traditional intercropping systems in a modern agricultural context:
%1. Agronomic Constraints (Competition and Yield Inefficiency)
%A primary disadvantage of growing multiple species together is the potential for competition and reduced output efficiency compared to specialized cultivation:
%• Compromised Crop Growth: Intercropping can lead to significant competition, particularly when one crop is fast-growing and aggressive. The foliage of Mauka (Mirabilis expansa), due to its decumbent nature, grows and expands rapidly, thus compromising the growth potential other species when cultivated in association. Farmers reported that close intercropping compromised root production for mauka.
%• Yield Reduction in Specialized Production: In integrated systems that prioritize multiple outputs (like food and fodder), maximizing one output can negatively affect the primary yield component. For achira (Canna indica), the leaf mass production provides environmental benefits and fodder, but increasing leaf yield can reduce starch yield.
%• Negative Density Effects: While high diversity indices characterize traditional farming systems, high plant density (sometimes associated with mixed systems on small plots) resulted in low root yield for ahipa landraces in a study, potentially as a result of low soil fertility. Conversely, increasing plant density in ahipa had a negative effect on root and pod growth per plant.
%2. Labor and Management Demands
%Traditional intercropping systems often rely on specific, labor-intensive practices that are disadvantageous in modern systems where labor costs are high and alternative livelihoods exist:
%• Intensive Manpower Requirement: Traditional cropping of the root legume ahipa requires a laborious yield-enhancing practice: reproductive pruning (manual removal of flowers or reproductive shoots) to redirect assimilates to the tuberous root. Although pruning dramatically increases root yield (more than five times for some accessions), this operation requires intensive manpower.
%• High Production Costs: The complexity and labor demands of traditional polyculture are becoming economically unviable. Priority is increasingly given to short growth-cycle cash crops that are less labour demanding. Commercial crops are replacing ahipa cultivation because of their quick economic returns and because they require less care.
%• Cumulative Labor Demands: The shift toward modern agriculture is driven by the fact that traditional crops like ahipa require a lot of labor, making them less competitive than marketable crops that require less labor.
%3. Market and Modernization Pressure
%The inherent nature of intercropping often clashes with the demands of modern markets and industrialization, leading farmers to abandon these practices in favor of monoculture:
%• Threat of Monoculture: The conservation of on-farm agro-biodiversity, typically maintained through mixed cropping, is threatened by monoculture and marketable crops. Farmers are increasingly opting for monocrops and crops with short growth cycles.
%• Lack of Standardization: Traditional farming systems are based on small land sizes, and the resulting production volume and variability are often inconsistent for modern market demands, leading to a low market value for crops like ahipa.
%• Loss of Knowledge: The dynamics of change cause the traditional cropping systems to rapidly disappear, along with the associated ethno-ecological knowledge required to manage complex intercropping systems effectively.
%Examples of Situations Where Intercropping May Not Be the Best Strategy
%Intercropping is demonstrated to be suboptimal when the goals shift toward commercial efficiency, industrial production, or risk mitigation related to specific crop characteristics:
%1. Industrial Processing Requiring Large, Uniform Volumes:
%    ◦ For the industrial processing of ahipa roots, the high cost of manpower associated with manual reproductive pruning (necessary in many traditional systems) could be avoided by selecting low flowering landraces and/or increasing planting density. This implies that a specialized, dense monoculture would be necessary to avoid the labor constraint inherent in the polyculture approach if the objective is competitive industrial raw material production.
%    ◦ Similarly, for starch extraction from achira, if maximizing starch output is the goal, maximizing leaf yield for fodder (a benefit in intercropping) must be avoided, favoring systems focused solely on rhizome production.
%2. When Specific Competitive Crops are Involved:
%    ◦ Intercropping mauka with other species may not be the best strategy for maximizing the yield of the companion crop due to mauka's tendency for rapid foliage expansion which compromises the growth potential of others. Farmers sometimes plant Mauka along the edge of the main crops, rather than closely intercropped, mitigating this competitive disadvantage while retaining some level of integration.
%3. In Environments with Intensified Land Use and High Economic Priority:
%    ◦ In areas where intensified land use is adopted, such as the lower altitudes of the East Andean valley in Bolivia and Manabí, Ecuador, monocropping is the rule and this intensive approach does not allow the practice of shifting cultivation (a traditional form of rotational mixed cropping). Farmers choose this pathway because marketable crops provide quick economic returns, suggesting that intercropping is suboptimal when short-term profitability and land efficiency are paramount.
%4. When Managing Land-Use Conflicts (Agropastoral Systems):
%    ◦ In the Bolivian Altiplano, the expansion and intensification of quinoa cultivation for cash cropping created a direct competition for land use with traditional llama husbandry. While quinoa was traditionally cultivated alongside llama husbandry in a balanced rotation, the intensified, specialized focus on quinoa (monoculture trend) led to llamas grazing on quinoa plants and pastoralists being forced to pasture their animals in further marginalized areas. This demonstrates that specialized, high-intensity systems supplant mixed systems when economic incentives shift, creating a land-use conflict where the mixed system is no longer maintained.


\newpage
\section{Question 5 - Seed Germination}
\textbf{Germination test of three seed lots of cowpea supplied by a farmer }

\begin{figure}[h]
    \centering
    \includegraphics[width=0.35\textwidth]{Figures/tcp_q_05.JPG}
    \caption{The relation between percent germinated seed over time of three cowpea seed lots.}
    \label{fig:tcp_q_05}
\end{figure}

\begin{enumerate}
    \item \textbf{Describe the curves.}

    All three curves represent cumulative germination over time. Curve \textbf{a} starts early and rises steeply, indicating high vigour and uniform germination. Curve \textbf{b} begins later and increases more gradually but eventually reaches a similar final level, suggesting viable seeds with lower vigour or mild dormancy. Curve \textbf{c} rises slowly and never reaches full germination, reflecting aged or partially dormant seeds with reduced viability.

    \vspace{0.5em}
    \item \textbf{Are they as expected for an ordinary germination test?}

    Yes. Under standard ISTA test conditions (stable temperature, light, and moisture), cumulative germination typically follows a sigmoidal pattern. Seeds germinate asynchronously due to individual physiological variation, and differences among lots mainly appear in onset time, germination rate, and final percentage.

    \vspace{0.5em}
    \item \textbf{How to estimate the germination curve mathematically?}

    Germination-time data can be modelled using the cumulative log-logistic function:
    \[
    F(t) = \frac{1}{1 + \left(\frac{t}{t_{50}}\right)^{b}}
    \]
    where \(F(t)\) is the cumulative fraction germinated at time \(t\). This model captures the sigmoidal curve shape and allows quantitative comparison of germination dynamics among seed lots.

    \vspace{0.5em}
    \item \textbf{Necessary parameters in the model.}

    The main parameters are:
    \begin{itemize}
        \item \(t_{50}\): Time required for 50\,\% germination, representing median germination rate and vigour.  
        \item \(b\): Shape parameter defining curve steepness and synchrony of germination.  
        \item Upper asymptote: Maximum germination percentage achieved by the lot.  
    \end{itemize}

    \vspace{0.5em}
    \item \textbf{How to test whether seed lots a and b differ significantly?}

    Fit the log-logistic model separately for each seed lot and compare parameter estimates (\(t_{50}\), \(b\), and maximum \%). Statistical significance can be assessed by confidence-interval overlap or ANOVA on model parameters. A lower \(t_{50}\) indicates faster germination and higher seed vigour.

    \vspace{0.5em}
    \item \textbf{Why does seed lot c have another shape than a and b?}

    Curve \textbf{c} shows a lower final germination and flatter slope, typical of heterogeneous or damaged seed lots. Reduced performance may result from ageing, mechanical damage, or partial dormancy, all of which limit water uptake and delay germination even under optimal test conditions.

    \vspace{0.5em}
    \item \textbf{If lots a and b are identical but tested under different conditions, what could explain curve b?}

    Curve \textbf{b} could represent results from a seedling emergence test in soil rather than a standard paper germination test. In soil, mechanical resistance, fluctuating temperature, and reduced oxygen slow germination and broaden the response curve, producing a delayed and flatter pattern even for identical seed material.
\end{enumerate}


%Intercropping is characterized as the practice of cultivating multiple crops together within the same plot, frequently associated with traditional farming systems or polyculture systems utilized by indigenous smallholder farmers in regions such as the Inter-Andean Valleys (IAV) and the Bolivian Altiplano. These traditional farming systems, such as chiru, ananta, and k'ata, are dynamic units that have historically enabled farmers to maintain a high agro-biodiversity. Examples include the cultivation of the root legume ahipa (Pachyrhizus ahipa) in polyculture systems alongside crops such as tomato, onion, fennel, arracacha, groundnut, cassava, white giant maize corn, and achoccha. Likewise, mauka (Mirabilis expansa) is frequently intercropped with maize, beans, yacón, and squash, and other Andean crops like oca, ulluco, and mashua are traditionally planted together.
%The practice of intercropping provides several potential benefits, primarily centered on sustainability, resource optimization, and resilience. One major benefit is the integral role of legumes like ahipa in promoting a sustainable land-use system by acting as N2 -fixing crops, which allows them to be grown without N fertilizers. This feature, along with pest tolerance, reduces input requirements and lowers environmental impact and production costs. Furthermore, the combination of a tuberous root crop and a legume can offer the yield reliability of a tuberous root crop alongside the high sustainability of a legume. These systems allow farmers to obtain multiple outputs; for instance, dried hay from Pachyrhizus is used as fodder, provided reproductive pruning is implemented to avoid toxic rotenone. Certain crops within these agro-ecological systems also provide protection, as seen with mauka, which synthesizes ribosome-inactivating proteins that have an antimicrobial effect against root-rotting microorganisms and generally does not succumb to pests or diseases.
%Despite these benefits, intercropping systems face significant challenges and potential disadvantages in modern agricultural contexts, often leading to their decline in favor of marketable crops. The foremost constraint relates to labor and economic viability; traditional methods require intensive manpower. For ahipa, the laborious yield-enhancing practice of reproductive pruning (manual removal of flowers) is needed to increase root yield. Conversely, farmers are increasingly giving priority to short growth-cycle cash crops that are less labour demanding, such as cereals (Poaceae) and vegetables (Solanaceae). This low market value endangers conservation of crops like ahipa, which is associated with traditional farming systems involving ethno-ecological knowledge.
%Intercropping may also be a suboptimal strategy when facing certain agronomic challenges or when aiming for commercial efficiency. Competition between associated crops is a recognized issue. For example, the foliage of mauka, due to its decumbent nature, grows and expands rapidly, thus compromising the growth potential other species when cultivated in association. Consequently, farmers have reported that close intercropping compromised root production. In situations where industrial utilization is prioritized, monoculture may be more advantageous; for instance, the high cost of manual pruning for ahipa could be avoided by breeding low flowering landraces and/or increasing planting density in specialized production systems. The viability of intercropping is also severely challenged by land-use conflict; the expansion of quinoa production for cash cropping, often displacing traditional mixed fields, directly competes with and endangers the robust agropastoral ecosystems of the Bolivian Altiplano. Site 1 in one study reported that the most common limitation to ensuring sufficient pasture land was that the land was already in use for quinoa production, leading to a significantly lower perception of sufficient pasture land available.

\newpage
\section{Question 6 - Sugar Production}
\begin{enumerate}
    \item \textbf{What is sugar?}  
    \begin{itemize}
        \item Sugar refers to simple carbohydrates such as sucrose, glucose, and fructose that act as primary energy sources for both plants and humans. In many root and tuber crops, sweetness arises from the enzymatic hydrolysis of complex carbohydrates or fructooligosaccharides (FOS) into these simple sugars, a process often promoted by sun exposure and physiological maturation.
    \end{itemize} 

    \item \textbf{Mention all the crops you known which are used for sugar production.}  
    \begin{itemize}
        \item Crops include yacon, oca, mashua, ahipa, achira, and sugarcane. Their main sugars are sucrose, glucose, and fructose in varying proportions. Yacon and oca are used for syrup or natural sweetener production, while sugarcane is the primary industrial source of sugar, molasses, and syrup worldwide.
    \end{itemize} 

    \item \textbf{Sugar cane can be used for several purposes. Mention these.}  
    \begin{itemize}
        \item Sugarcane provides multiple products: molasses (miel de caña), concentrated syrup (chancaca), crystalline sugar, and fermented beverages such as ethanol. The fibrous residue (bagasse) serves as fuel or livestock feed, and fresh cane juice or mixed by-products are often used in traditional foods or as animal fodder with crops like sweetpotato vines.
    \end{itemize}

    \item \textbf{Explain how sugar cane normally is established in the field.} 
    \begin{itemize}
        \item Sugarcane is propagated vegetatively using stem cuttings called setts, each containing 2-3 buds. These are planted horizontally or at slight angles in furrows, then covered lightly with soil. Good soil moisture and temperature favour sprouting and tiller formation. This clonal establishment ensures uniformity and maintains selected varieties.
    \end{itemize}

    \item \textbf{Is it necessary to establish a new sugar cane crop each year?}  
    \begin{itemize}
        \item No. After the first planting, called the plant crop, new shoots arise from underground buds of the previous stools, producing one or more ratoon crops. Replanting occurs only when yield or quality declines significantly-often after 3-5 harvests depending on soil fertility and management.
    \end{itemize}

    \item \textbf{Why and when do farmers many places in the world ignite sugar cane fields?} 
    \begin{itemize}
        \item Farmers often burn mature cane fields before harvest to remove dry leaves, facilitate manual cutting, and reduce pest habitat. Burning may also be used after harvest to clear residues. However, the practice is declining due to environmental regulations, nutrient loss, and air pollution concerns.
    \end{itemize}

    \item \textbf{After sugar canes have been processed in a factory, which types of waste products are produced and what can they be used for?} 
    \begin{itemize}
        \item Main by-products include bagasse, molasses, and filter cake. Bagasse is used as boiler fuel or for paper and fibreboard; molasses is a substrate for ethanol or rum production; and filter cake, rich in organic matter and phosphorus, serves as fertiliser or compost material, contributing to circular resource use.
    \end{itemize}
\end{enumerate}

\newpage
\section{Question 7 - Cropping Systems}

A small subsistence farm is placed in a semi-arid area in the highlands of Guatemala (500 mm of annual precipitation). Crops are grown in rotation with a short fallow period, and fertilizer is not used on the subsistence crops. The field is not irrigated. Answer the following questions for a field grown with maize intercropped with bean. 

Rainy season, sowing, and harvesting

In the semi-arid highlands of Guatemala, annual rainfall averages about 500 mm, concentrated between May and October, followed by a pronounced dry season. Farmers time sowing to coincide with the onset of the rains in May-June. Under rain-fed conditions, maize requires roughly 110-150 days to maturity and is harvested around September-October, whereas beans mature in 50-70 days and are harvested in July-August, often before maize reaches flowering.

Estimated yield for maize monocrop

A realistic yield for an unfertilised, rain-fed maize monocrop in these semi-arid highlands is about 1 t ha$^{-1}$, depending on rainfall distribution, soil fertility, and management.

Yield improvement with intercropping

Intercropping with bean can modestly raise total system productivity. Beans fix atmospheric nitrogen through symbiosis with Rhizobium, enriching soil N and benefiting the maize component. Root systems differ in depth, improving water and nutrient partitioning, while canopy interaction reduces evaporation and soil erosion. Moreover, the system buffers climatic risk-if drought affects one crop, the other may still yield.

Crop establishment

The field is ploughed at the end of the dry season using oxen or hand tools to conserve residual moisture. Maize is sown in rows at the onset of rain, with beans interplanted between rows or after maize emergence to avoid competition. Residue mulching and contour planting help retain soil moisture and prevent erosion on sloping land.

Advantages of intercropping maize and bean
\begin{enumerate}
    \item Biological nitrogen fixation enhances soil fertility and sustainability.
    \item Improved land equivalent ratio through complementary resource use.
    \item Reduced drought risk and greater yield stability.
    \item Structural support: maize provides stakes and partial shade for climbing beans.
    \item Maintenance of agrobiodiversity typical of traditional milpa systems, sustaining ecological and cultural resilience.
\end{enumerate}

\newpage
\section{Question 8 - Cropping Systems}
\textbf{Quinoa in Bolivia:}

\vspace{0.5em}
A farmer close to Titicaca Lake in Bolivia grows quinoa as one of his main crops. His village receives about 800 mm rain per year. 

\vspace{0.5em}
\textbf{How is rainfall and temperature distributed over the year? }
In the highlands around Lake Titicaca, annual rainfall averages about 800 mm, falling mainly between September and April. The remaining months, May to August, form a pronounced dry season with very little precipitation. Temperatures fluctuate sharply due to altitude: daytime maxima reach 17-19 \textdegree C, while night minima often fall to 0-3 \textdegree C, occasionally below freezing. The mean cropping-season temperature of 9-10 \textdegree C suits quinoa’s physiology, as it germinates and grows well near its base temperature of about 3 \textdegree C, showing strong tolerance to cold and frost.

\vspace{0.5em}
\textbf{When is quinoa sown and harvested? }
Quinoa is sown from September to November at the onset of the rains, ensuring moisture for germination and early vegetative growth. The crop matures toward the end of the wet season and is harvested in April or May, as rainfall declines and grain drying conditions improve.

\vspace{0.5em}
\textbf{Which other crops may the farmer grow? }
Farmers near Lake Titicaca typically combine quinoa with other native Andean crops adapted to cool, semi-humid highland conditions. Common rotations or companion crops include potatoes (Solanum spp.), cañahua (Chenopodium pallidicaule), and root and tuber crops such as oca, mashua, and ulluco. These species thrive within the 700-1000 mm rainfall range and share similar temperature tolerance. The system is often agropastoral, integrating llamas or alpacas whose manure maintains soil fertility and supports a resilient mixed-farming livelihood typical of the Altiplano.

\newpage
\section{Question 9 - Intercropping}

\begin{enumerate}
    \item \textbf{Discuss monocropping vs. intercropping. Mention at least 3 advantages/3 disadvantages for each.}

    Monocropping is the cultivation of a single crop species over a field or area, while intercropping (or polyculture) involves growing two or more species simultaneously on the same land.

    \vspace{0.5em}
    \textbf{Monocropping}

    \begin{table}[h]
        \centering
        \caption{Examples of advantages and disadvantages of monocropping.}
        \label{tab:monocropping_adv_disadv}
        \rowcolors{2}{white}{gray!7}
        \begin{tabular}{p{5cm}|p{5cm}}
        \textbf{Advantages} & \textbf{Disadvantages} \\
        \hline
        Simplicity in management, fertilisation, and mechanisation. & Low agrobiodiversity increases vulnerability to pests and diseases. \\
        Uniform pest and disease control strategies. & Soil nutrient depletion and reduced long-term sustainability. \\
        High yield potential and economic specialisation for a single crop. & High economic risk - total loss if the crop fails. \\
        \end{tabular}
    \end{table}

    \vspace{0.5em}
    \textbf{Intercropping}

    \begin{table}[h]
        \centering
        \caption{Examples of advantages and disadvantages of intercropping.}
        \label{tab:intercropping_adv_disadv}
        \rowcolors{2}{white}{gray!7}
        \begin{tabular}{p{5cm}|p{5cm}}
        \textbf{Advantages} & \textbf{Disadvantages} \\
        \hline
        Enhanced resource-use efficiency through complementary rooting and canopy structures. & Increased labour and management complexity. \\
        Biological nitrogen fixation by legumes improves soil fertility. & Competition for light, water, or nutrients may reduce individual crop yields. \\
        Reduced production risk and improved system resilience. & Limited mechanisation and market standardisation in large-scale systems. \\
        \end{tabular}
    \end{table}

    \item \textbf{How would you plan a trial to test the effect of intercropping?}

    A randomised complete block design (RCBD) or split-plot design can be used with the following treatments:
    \begin{itemize}
        \item Monocrop A (e.g., maize)
        \item Monocrop B (e.g., bean)
        \item Intercrop A + B (alternate rows or mixed plots)
    \end{itemize}

    All plots should maintain equal plant densities and similar soil conditions. Replication is essential to account for spatial variability. Data should include yield and yield components for each crop, soil nitrogen, and microclimatic factors.

    Statistical analysis can be performed using ANOVA followed by Tukey’s HSD test to detect significant differences among treatments. For multi-environment studies, AMMI or GGE biplot analysis can evaluate genotype $\times$ environment interactions.

    \vspace{0.5em}
    \item \textbf{What is Land Equivalent Ratio (LER)?}

    The Land Equivalent Ratio (LER) quantifies land-use efficiency of intercropping compared with monocropping:
    \[
    LER = \frac{Y_{ab}}{Y_a} + \frac{Y_{ba}}{Y_b}
    \]
    where $Y_{ab}$ and $Y_{ba}$ are intercrop yields of each species, and $Y_a$, $Y_b$ are monocrop yields.

    \vspace{0.5em}
    Interpretation:
    \begin{itemize}
        \item $LER > 1$: intercropping uses land more efficiently than monocropping.
        \item $LER = 1$: equal efficiency.
        \item $LER < 1$: disadvantage compared with monocropping.
    \end{itemize}

    A high LER indicates complementary resource use and higher overall productivity per unit area.
\end{enumerate}


%\begin{enumerate}
%    \item Discuss monocropping vs intercropping. Mention at least 3 advantages/ 3 disadvantages for each. 
%        \begin{table}[h]
%            \centering
%            \caption{A table with some examples of both advantages and disadvantages of monocropping.}
%            \label{tab:table_q_09.1}
%            \rowcolors{2}{white}{gray!7}
%            \begin{tabular}{p{5cm}|p{5cm}}
%            \textbf{Advantages} & \textbf{Disadvantages} \\
%            \hline
%            1. Simplicity in management and mechanization. & 1. Increased complexity in management. \\
%            2. Easier to apply uniform pest and disease control. & 2. Potential for increased competition between crops. \\
%            3. Specialization can lead to higher yields of a single crop. & 3. Risk of total crop failure due to pests or diseases. \\
%            \end{tabular}
%        \end{table}

%    2. What is intercropping? \begin{table}[h]
%            \centering
%            \caption{A table with some examples of both advantages and disadvantages of intercropping.}
%            \label{tab:table_q_09.2}
%            \rowcolors{2}{white}{gray!7}
%            \begin{tabular}{p{5cm}|p{5cm}}
%            \textbf{Advantages} & \textbf{Disadvantages} \\
%            \hline
%            1. Simplicity in management and mechanization. & 1. Increased complexity in management. \\
%            2. Easier to apply uniform pest and disease control. & 2. Potential for increased competition between crops. \\
%            3. Specialization can lead to higher yields of a single crop. & 3. Risk of total crop failure due to pests or diseases. \\
%            \end{tabular}
%        \end{table}
    
%    \item How would you plan a trial to test the effect of intercropping?  
%    \item What is Land Equivalent Ratio?
%\end{enumerate}

\newpage
\section{Question 10 - Fertility of Tropical Soils}

\begin{enumerate}
    \item \textbf{Give an overview of the benefits of increasing the content of organic carbon in soil.}

    Higher soil organic carbon improves the physical, chemical, and biological properties of soil. It enhances soil structure, porosity, and water-holding capacity-key factors in semi-arid tropical systems where moisture retention determines yield stability. Chemically, organic carbon increases nutrient-holding capacity and cation exchange, reducing nutrient losses through leaching. Biologically, it stimulates microbial activity, enzyme production, and nutrient cycling. In tropical regions, high temperature and rainfall accelerate decomposition, so maintaining organic carbon is essential to prevent erosion, preserve aggregates, and buffer nutrient fluctuations. Crops such as \textit{mauka} perform best in soils rich in organic matter ($\ge$3\%), showing improved growth and productivity under these conditions.

    \vspace{0.5em}
    \item \textbf{What is the cation exchange capacity (CEC) of a soil, and how does it affect soil fertility?}

    Cation exchange capacity (CEC) expresses the soil’s ability to hold and exchange positively charged ions such as K$^+$, Ca$^{2+}$, and Mg$^{2+}$. Soils with high CEC retain nutrients longer and supply them more effectively to plants, maintaining fertility throughout the cropping cycle. Organic matter and clay minerals contribute most to CEC. In highly weathered tropical soils such as Oxisols and Ultisols, low clay activity means fertility depends largely on the organic matter fraction. Increasing soil organic carbon therefore directly supports nutrient retention, buffering capacity, and long-term yield stability.

    \vspace{0.5em}
    \item \textbf{Explain how the content of organic carbon of a soil can be increased.}

    Organic carbon can be increased by adding organic amendments such as manure, compost, or humus, and by integrating biomass-producing or nitrogen-fixing crops like \textit{achira} and \textit{ahipa}. Agropastoral practices, such as llama husbandry, recycle nutrients through manure return to the fields. Incorporating crop residues, cover crops, and diversified rotations further builds soil organic matter, improving fertility and sustainability. Because decomposition rates are high in the tropics, maintaining SOM requires continuous inputs from crop residues and organic fertilizers. Integrating livestock closes nutrient loops typical of agropastoral systems, ensuring long-term soil health and crop resilience under tropical conditions.
\end{enumerate}

\newpage
\section{Question 11 - Legumes as Soil Nutrients Providers}
\begin{enumerate}
    \item \textbf{Explain how legumes can play an important role in tropical production systems.}

    Legumes such as \textit{Pachyrhizus ahipa} are central to tropical production through biological nitrogen fixation, forming symbioses with \textit{Rhizobium} or \textit{Bradyrhizobium} that replace synthetic N fertilisers. They return large quantities of nitrogen-rich residues to the soil (up to 215 kg N ha$^{-1}$), improving soil fertility, reducing input costs, and supporting low-input sustainability. In nutrient-poor, highly weathered tropical soils, this process replenishes N stocks and enhances soil organic matter. Their adaptability to smallholder systems and contribution to agrobiodiversity make legumes key to resilient, climate-adapted agriculture.

    \vspace{0.5em}
    \item \textbf{What is the difference between a legume green manure and a legume cover crop?}

    Both improve soil fertility but differ in management purpose. A green manure is grown mainly to be incorporated into the soil, adding organic matter and nitrogen. A cover crop protects the surface from erosion, suppresses weeds, and can provide fodder before decomposition. In \textit{Pachyrhizus} species, the above-ground biomass can serve either role-incorporated as green manure or harvested as fodder-while nutrients re-enter the system through manure recycling.

    \vspace{0.5em}
    \item \textbf{Are legumes always an advantage for the following crop in the crop rotation?}

    Not necessarily. While legumes leave a positive nitrogen balance, continuous cultivation can favour pest, disease, or nematode buildup. For example, \textit{Pachyrhizus erosus} yields decline after two consecutive seasons and requires 3–4 years of rotation to restore soil and pest balance. Thus, legumes provide maximum benefit only when integrated into diverse, well-managed rotations that maintain biological and soil health.

    \vspace{0.5em}
    \item \textbf{Explain how nitrogen fertiliser interacts with the ability of a legume to fix nitrogen.}

    External nitrogen supply suppresses symbiotic fixation because plants preferentially absorb available mineral N instead of fixing atmospheric N$_2$. In \textit{ahipa}, inoculation with efficient rhizobia alone achieved full yield potential, confirming that additional N fertiliser is unnecessary. High fixation rates (up to 215 kg N ha$^{-1}$) demonstrate that these legumes can meet their N requirements independently, reducing costs and environmental impacts while sustaining fertility in tropical soils.
\end{enumerate}

\newpage
\section{Question 12 - Fertilizers and Manure}
\begin{enumerate}
    \item \textbf{What are the advantages and disadvantages of using chemical fertilizers in tropical production systems?}

    \textbf{Advantages:}
    \begin{itemize}
        \item Provide high nutrient concentrations that are rapidly available to crops, giving quick yield responses.  
        \item Enable precise nutrient management and short-term yield maximisation; in \textit{mauka}, chemical fertilisers produced up to 78.5\,t\,ha$^{-1}$ roots.  
        \item Essential for intensive systems where nutrient mining and high crop demand require immediate replenishment.
    \end{itemize}

    \textbf{Disadvantages:}
    \begin{itemize}
        \item Continuous use without organic inputs accelerates soil acidification, structure decline, and loss of biological activity.  
        \item High leaching risk under tropical rainfall and low CEC soils leads to poor nutrient-use efficiency and water pollution.  
        \item Creates economic dependency and masks underlying management issues such as low organic matter or erosion.
    \end{itemize}

    \vspace{0.5em}
    \item \textbf{Compare these with organic fertilizers and explain advantages and disadvantages of these.}

    \textbf{Advantages:}
    \begin{itemize}
        \item Improve soil structure, porosity, and moisture retention-key for semi-arid and weathered tropical soils.  
        \item Increase cation exchange capacity and stimulate microbial life, enhancing nutrient cycling and long-term fertility.  
        \item Recycle nutrients within the farm system; manure and compost supply N, P, and K while raising soil organic carbon.
    \end{itemize}

    \textbf{Disadvantages:}
    \begin{itemize}
        \item Lower immediate yields than chemical fertilisers (often about 60 \% of chemically fertilised yields).  
        \item Limited local availability and competing household uses constrain large-scale application.  
        \item Nutrient release is slow and depends on decomposition rate, requiring continuous management for cumulative effect.
    \end{itemize}

    \vspace{0.5em}
    \item \textbf{Explain the importance of synchrony of supply and demand for nitrogen.}

    Nitrogen must be supplied when crop uptake is highest to avoid both deficiency and leaching loss. In \textit{ahipa}, synchrony is improved through reproductive pruning, which redirects assimilates and N to root development. Applying manure before planting ensures mineralisation during early growth, while legumes maintain synchrony naturally through symbiotic fixation that releases N in pace with plant demand. Poor synchrony in tropical soils-where mineralisation is rapid-results in low nitrogen-use efficiency.

    \vspace{0.5em}
    \item \textbf{What is a nutrient deficiency symptom in plants, and what can be learnt from them?}

    Nutrient deficiency symptoms-such as chlorosis, stunted growth, or poor yields-reflect imbalances in soil fertility. They indicate which nutrients are limiting productivity and guide corrective management. In tropical systems, analyses show that N, P, or K deficiencies sharply reduce root yields, while balanced nutrition enhances both yield and nutritional quality. Observing these symptoms helps diagnose soil constraints and refine fertiliser or crop-rotation strategies for sustainable production.
\end{enumerate}

\newpage
\section{Question 13 - The Importance of Agrobiodiversity}

\begin{enumerate}
    \item \textbf{Mention four reasons why agro-biodiversity matters for crop breeding.}
        \begin{enumerate}
            \item Provides essential \textbf{genetic resources for stress tolerance}, including frost- and drought-resistant landraces of \textit{quinoa}, \textit{cañahua}, and native \textit{potatoes}. Such traits are crucial for breeding climate-resilient crops under increasing environmental variability.
            \item Enhances \textbf{nutritional and functional diversity} through traits like high mineral and vitamin content or unique bioactive compounds (e.g. fructooligosaccharides in \textit{yacon}, glucosinolates in \textit{mashua}). These traits expand breeding targets beyond yield to include health-promoting quality.
            \item Offers \textbf{natural pest and disease resistance}, reducing dependence on pesticides. For instance, \textit{mauka} produces antimicrobial proteins that can be introgressed into breeding lines to strengthen resistance.
            \item Supplies valuable \textbf{agronomic and quality traits}, such as improved starch characteristics, root size, or specific fruit qualities useful for food industry and processing. Preserving these alleles widens breeding options and supports innovation.
        \end{enumerate}

    \vspace{0.5em}
    \item \textbf{What are the challenges to work with it? Give at least five examples.}
        \begin{enumerate}
            \item Economic pressure from short-cycle, high-value commercial crops promotes monocultures and discourages local diversity.  
            \item Loss of traditional knowledge as younger generations migrate, leading to erosion of seed selection and cultivation skills.  
            \item Cultural stigma labelling native crops as “poor people’s food,” reducing consumption and market demand.  
            \item Agronomic limitations such as seed shattering in \textit{cañahua} or high labour demand in \textit{ahipa} limit adoption.  
            \item Biopiracy and restrictive access regulations under the Nagoya Protocol complicate equitable use of genetic resources.  
            \item Insufficient ex situ conservation and limited research investment for underutilised crops like \textit{mauka}, leaving genetic material at risk.  
        \end{enumerate}

    \vspace{0.5em}
    \item \textbf{How to conserve this agro-biodiversity?}
        \begin{enumerate}
            \item \textbf{In situ conservation:} Maintain crop diversity on-farm by supporting farmers as biodiversity custodians, ensuring continued cultivation and adaptive selection of landraces under local conditions.  
            \item \textbf{Ex situ conservation:} Collect and preserve germplasm in national and international seed banks, facilitating research and breeding access.  
            \item \textbf{Market revalorisation and policy support:} Promote native crops by developing value-added products (e.g. \textit{yacon} syrup, \textit{mashua} flour), improving market access, and providing training in processing and branding. Strengthening farmer organisations and linking conservation to economic incentives ensures long-term sustainability and prevents genetic erosion.  
        \end{enumerate}
\end{enumerate}

\newpage
\section{Question 14 - Crop Phenotyping}

\begin{enumerate}
    \item \textbf{How can phenotyping approaches support crop production in the future?}

    Phenotyping enables precise and high-throughput measurement of plant traits to identify and improve genotypes best suited to future climates and management systems. It supports breeding by quantifying adaptive traits such as drought, heat, and frost tolerance; improves agronomic efficiency by analysing growth responses to temperature, light, and density; and supports modelling of thermal time and base temperature to optimise sowing and harvest scheduling. By linking physiological and biochemical traits to productivity, phenotyping also contributes to improved nutritional quality, resilience, and resource-use efficiency-key goals in sustainable tropical agriculture. Emerging digital and automated tools further allow integration of phenotypic data with genomics to accelerate breeding for climate adaptation.

    \vspace{0.5em}
    \item \textbf{Can you give some examples of applications in the tropics?}
    \begin{itemize}
        \item \textbf{\textit{Cañahua:}} Germination and seedling phenotyping across 3–24\,\textdegree C and varying sowing depths identifies landraces with rapid emergence and cold-soil adaptation.  
        \item \textbf{\textit{Ahipa:}} Root growth and yield phenotyping relate productivity to accumulated heat units and temperature, guiding site selection and planting time.  
        \item \textbf{\textit{Capsicum} and ARTCs:} Chemical phenotyping detects accessions rich in flavonoids, FOS, and glucosinolates, supporting breeding for nutritional quality and functional foods.  
        \item \textbf{Participatory sensory phenotyping:} Evaluates colour, flavour, and texture with local communities to align crop improvement with consumer preferences and cultural identity.  
    \end{itemize}

    \vspace{0.5em}
    \item \textbf{Do you see difficulties in its application in the tropics? How can they be overcome?}

    Tropical phenotyping faces major challenges: high environmental variability, limited infrastructure, genetic heterogeneity, and low commercial value of many native crops.  
    \begin{itemize}
        \item Apply multivariate and modelling approaches to handle genotype $\times$ environment interactions under field conditions.  
        \item Combine quantitative phenotyping with farmers’ experiential knowledge through participatory approaches to capture locally relevant traits.  
        \item Standardise morphological descriptors and strengthen germplasm conservation to ensure data comparability across sites.  
        \item Link phenotyping results to value-chain development, gastronomy, and market innovation to revalue traditional crops and ensure research impact.  
        \item Invest in low-cost, portable, or image-based tools adapted to tropical environments to overcome logistical constraints.  
    \end{itemize}
\end{enumerate}

\newpage
\section{Question 15 - Small and Large Scale Farming}

\begin{enumerate}
    \item \textbf{Mention at least two characteristics for each: small- and large-scale farming systems.}

    \textbf{Small-scale farming:}
    \begin{itemize}
        \item High agrobiodiversity on limited land, relying on mixed or intercropping systems such as \textit{chiru} and \textit{ananta}, which integrate crops and livestock.  
        \item Low external input dependence, using traditional tools, organic manure, and local ecological knowledge to maintain soil fertility and risk resilience.  
    \end{itemize}

    \textbf{Large-scale farming:}
    \begin{itemize}
        \item Dominated by monocultures of market-oriented crops with high mechanisation and capital investment; for example, quinoa expansion on the Altiplano.  
        \item Focused on maximising yield and profit, often competing with traditional agropastoral systems such as llama husbandry and reducing biodiversity.  
    \end{itemize}

    \vspace{0.5em}
    \item \textbf{Define “sustainable intensification” and explain why some people consider the term self-contradictory.}

    Sustainable intensification seeks to increase food production while minimising environmental impact and preserving ecosystem services. It emphasises closing yield gaps through improved agronomic efficiency rather than higher input use.  

    Some critics see the concept as self-contradictory because historical intensification has caused biodiversity loss, soil degradation, and greenhouse gas emissions. The term combines two opposing goals: “sustainability,” which implies ecological balance and low input use, and “intensification,” which implies higher productivity and resource extraction. However, advocates argue that through better management, diversification, and innovation, productivity and sustainability can coexist.

    \vspace{0.5em}
    \item \textbf{Give three examples of sustainable intensification and explain one of them in detail.}
        \begin{enumerate}
            \item \textbf{Legume-based systems:} Using \textit{Pachyrhizus} (e.g. \textit{ahipa}) to fix atmospheric nitrogen, reduce fertiliser need, and enrich soils.  
            \item \textbf{Integrated agropastoral systems:} Balancing llama grazing with quinoa cultivation to maintain nutrient cycling and livelihood stability.  
            \item \textbf{Improved agronomic management:} Optimising planting density, rotation, and residue management to raise yields sustainably.  
        \end{enumerate}

    \textbf{Detailed example:}  
    Tuberous legumes such as \textit{ahipa} form symbioses with \textit{Rhizobium} and \textit{Bradyrhizobium}, fixing up to 215\,kg\,N\,ha$^{-1}$ and eliminating the need for synthetic nitrogen. When residues are left on the field, they replenish soil fertility for subsequent crops. This system increases productivity while reducing external inputs, exemplifying sustainable intensification through biological nitrogen fixation, natural nutrient recycling, and minimal environmental impact. In tropical highlands, such practices improve both yield stability and long-term soil health, aligning productivity with sustainability goals.
\end{enumerate}


\newpage
\section{Question 16 - Fertilizer and Manure in the Tropics}

\begin{enumerate}
    \item \textbf{How to determine how much nutrients need to apply for crop growth? Discuss the practices to reduce nutrient losses and increase nutrient use efficiency.}
        \begin{enumerate}
            \item Nutrient needs in tropical systems are often established through field trials, soil quality targets, or biological autonomy. Empirical testing defines appropriate fertilizer or manure doses (e.g. 60 N and 40 P units, or 7.5 t ha$^{-1}$ manure). Crops like mauka require soils with $\ge$ 3 \% organic matter, while legumes such as ahipa rely on biological nitrogen fixation, eliminating external N requirements.
            
            \item To increase nutrient use efficiency (NUE) and reduce losses:
                \begin{enumerate}
                    \item Promote N fixation via legumes and rhizobial inoculation (fixing 58-215 kg N ha$^{-1}$).
                    \item Apply pruning in root crops to direct nutrients to the economic organ.
                    \item Recycle manure and residues from agropastoral systems every two years to replenish soil nutrients and organic carbon.
                \end{enumerate}
        \end{enumerate}

    \item \textbf{Discuss the advantages and disadvantages of using mineral and organic fertilizers.}
    \begin{table}[h]
        \centering
        \caption{Comparison of mineral and organic fertilizers in tropical systems.}
        \label{tab:table_q_16.2}
        \rowcolors{2}{white}{gray!7}
        \begin{tabular}{p{5cm}|p{5cm}|p{5cm}}
        \textbf{Fertilizer type} & \textbf{Advantages} & \textbf{Disadvantages} \\
        \hline
        Mineral Fertilizers (Chemical) & 
        \begin{itemize}
            \item High, immediate yield response (e.g. mauka 78.5 t ha$^{-1}$ roots).
            \item Concentrated nutrients easy to apply and control.
            \item Precise nutrient management for specific crop needs.
        \end{itemize} & 
        \begin{itemize}
            \item Does not improve soil structure or microbial health.
            \item Enhance cation exchange capacity and microbial activity.
            \item Recycle nutrients within the farming system.
        \end{itemize} \\
        Organic Fertilizers (Manure) & 
        \begin{itemize}
            \item Enhances physical (structure, porosity), chemical (CEC), and biological (microbial) properties.
            \item Recycles farm nutrients and supports sustainability.
            \item Provides N, P, and K naturally.
        \end{itemize} & 
        \begin{itemize}
            \item Produces lower yields ($\approx$ 59 \% of chemical fertilizer).
            \item Limited availability and labour-intensive handling.
            \item Often reserved for main crops like potato and maize.
            \item Slow nutrient release, requiring long-term application.
        \end{itemize} \\
        \end{tabular}
    \end{table}
\end{enumerate}

\begin{enumerate}
    \item \textbf{How to determine how much nutrients to apply for crop growth? Discuss practices to reduce nutrient losses and increase nutrient-use efficiency (NUE).}

    Nutrient requirements in tropical systems are established through field trials, soil fertility targets, and local empirical testing. Site-specific nutrient management defines appropriate fertiliser or manure rates-for example, 60\,kg\,N and 40\,kg\,P per hectare, or 7.5\,t\,ha$^{-1}$ manure. Crops such as \textit{mauka} perform optimally in soils with $\ge$\,3\,\% organic matter, whereas legumes like \textit{ahipa} rely on biological nitrogen fixation, removing the need for external N inputs.

    \vspace{0.5em}
    To increase NUE and reduce nutrient losses in tropical environments:
    \begin{itemize}
        \item \textbf{Promote biological N fixation:} Use legumes and rhizobial inoculation, fixing 58–215\,kg\,N\,ha$^{-1}$ and reducing fertiliser dependency.  
        \item \textbf{Optimise nutrient partitioning:} Apply pruning in root crops to redirect assimilates and nutrients to the economic organ.  
        \item \textbf{Recycle nutrients:} Reintegrate manure and crop residues from agropastoral systems every two years to replenish soil nutrients and organic carbon.  
        \item \textbf{Synchronise supply and demand:} Time fertiliser or manure applications with peak crop uptake to minimise leaching losses under heavy tropical rainfall.  
        \item \textbf{Improve soil structure:} Maintain soil cover and organic inputs to enhance infiltration and nutrient retention capacity.  
    \end{itemize}

    \vspace{0.5em}
    \item \textbf{Discuss the advantages and disadvantages of using mineral and organic fertilizers.}

    \begin{table}[h]
        \centering
        \caption{Comparison of mineral and organic fertilizers in tropical production systems.}
        \label{tab:table_q_16.2}
        \rowcolors{2}{white}{gray!7}
        \begin{tabular}{p{4cm}|p{5.5cm}|p{5.5cm}}
        \textbf{Fertilizer type} & \textbf{Advantages} & \textbf{Disadvantages} \\
        \hline
        \textbf{Mineral Fertilizers (Chemical)} &
        \begin{itemize}
            \item Provide high, immediate yield response (e.g. \textit{mauka} up to 78.5\,t\,ha$^{-1}$ roots).  
            \item Concentrated nutrients allow precise control and rapid crop correction.  
            \item Effective for short-term yield optimisation.  
        \end{itemize} &
        \begin{itemize}
            \item Do not improve soil structure, CEC, or microbial life.  
            \item High leaching risk under tropical rainfall and low CEC soils.  
            \item Long-term use can acidify soils and increase dependency.  
        \end{itemize} \\

        \textbf{Organic Fertilizers (Manure, Compost)} &
        \begin{itemize}
            \item Enhance soil structure, porosity, and water-holding capacity.  
            \item Improve cation exchange capacity and stimulate microbial activity.  
            \item Recycle farm nutrients, supplying N, P, and K naturally.  
            \item Support long-term sustainability and soil resilience.  
        \end{itemize} &
        \begin{itemize}
            \item Lower immediate yield response (often 50–60\% of chemical fertiliser yields).  
            \item Limited availability and labour-intensive handling.  
            \item Nutrient release depends on decomposition rate, requiring sustained management.  
        \end{itemize} \\
        \end{tabular}
    \end{table}

    \vspace{0.5em}
    \textbf{Summary:}  
    Chemical fertilizers maximise short-term yield but may degrade long-term soil quality, while organic fertilizers sustain fertility and structure but release nutrients slowly. Integrating both through combined or sequential applications improves nutrient-use efficiency, stabilises yields, and maintains soil health-key objectives for sustainable tropical crop production.
\end{enumerate}

\newpage
\section{Question 17 - Fertility of Tropical Soils}

\begin{enumerate}
    \item \textbf{What is soil fertility? How soil organic carbon helps to improve soil fertility?} 
        \begin{enumerate}
            \item Soil fertility is the soil's capacity to provide essential nutrients and physical conditions for healthy crop growth. In tropical and high-altitude regions, fertility determines crop establishment and yield potential under challenging conditions.
            \item Soil organic carbon (SOC), as part of organic matter, enhances fertility through:
                \begin{itemize}
                    \item Physical improvement: increases soil structure, porosity, and water retention, essential in semi-arid climates.
                    \item Chemical enrichment: boosts nutrient retention and cation exchange capacity.  
                    \item Biological enhancement: supports microbial activity that aids nutrient cycling.
                \end{itemize}
            \item Crops like mauka perform best in soils with $\ge$ 3 \% organic matter, showing SOC's key role in productivity.
        \end{enumerate}
    
    \item \textbf{What is the cation exchange capacity (CEC) of a soil? and how does CEC affect soil fertility?} 
        \item CEC is the soil's ability to hold and exchange positively charged nutrients (e.g. K$^+$, Ca$^{2+}$, Mg$^{2+}$) on clay and organic matter surfaces.
        \item A high CEC means:
            \begin{itemize}
                \item Better nutrient retention and reduced leaching.
                \item Greater availability of macronutrients like phosphorus and potassium.
                \item Enhanced chemical fertility, maintaining continuous nutrient supply for plants.
            \end{itemize}
        \item Thus, soils rich in organic matter and clay-those with high CEC-are more fertile, resilient, and productive under tropical conditions.
\end{enumerate}

\newpage
\section{Question 18 - Legumes as Soil Nutrients Providers}

\begin{enumerate}
    \item \textbf{What is the difference between a legume green manure and a legume cover crop? Discuss with examples, advantages and disadvantages of each method.}

    Green manure legumes are cultivated primarily to enrich the soil by incorporating their biomass directly into it.  
    \begin{itemize}
        \item \textbf{Example:} Leaving \textit{Pachyrhizus} (yam bean) tops in the field after harvest.  
        \item \textbf{Advantages:} Returns large quantities of fixed nitrogen to the soil, improving fertility, soil structure, and moisture retention while reducing external fertiliser needs.  
        \item \textbf{Disadvantages:} Biomass cannot be used as fodder or sold, limiting short-term economic returns and labour efficiency.  
    \end{itemize}

    Cover crops, in contrast, protect the soil surface from erosion, suppress weeds, and provide feed or mulch while indirectly adding nitrogen through residues or manure.  
    \begin{itemize}
        \item \textbf{Example:} Dried \textit{P. erosus} hay used as animal fodder mixed with maize or lucerne.  
        \item \textbf{Advantages:} Supplies protein-rich feed, prevents nutrient leaching during the rainy season, and recycles nutrients through manure.  
        \item \textbf{Disadvantages:} Potential rotenone toxicity if not properly managed and higher labour input for pruning and monitoring.  
    \end{itemize}

    Both methods are vital in tropical systems where high rainfall and erosion accelerate nutrient loss; green manures rebuild soil fertility, while cover crops maintain soil protection between production cycles.

    \vspace{0.5em}
    \item \textbf{What are the advantages and disadvantages of legumes in cropping systems?}

    \textbf{Advantages:}
    \begin{itemize}
        \item Fix atmospheric nitrogen (up to 215\,kg\,N\,ha$^{-1}$), lowering fertiliser costs and improving nutrient balance.  
        \item Enhance soil fertility, structure, and sustainability through rotation or intercropping.  
        \item Provide diverse outputs-protein-rich seeds, edible roots, and high-value fodder-supporting food security and income diversification.  
    \end{itemize}

    \textbf{Disadvantages:}
    \begin{itemize}
        \item Labour-intensive management (e.g., reproductive pruning and harvesting).  
        \item Presence of toxic compounds such as rotenone in unpruned plants limits utilisation.  
        \item Susceptibility to pest and nematode buildup requires long rotations of 3–4 years.  
    \end{itemize}

    \vspace{0.5em}
    \item \textbf{What are potential constraints in adaptation of legumes by smallholder farmers?}

    \begin{itemize}
        \item \textbf{Labour demands:} Manual pruning and biomass handling are time-consuming.  
        \item \textbf{Market limitations:} Low consumer awareness, weak demand, and poor transport reduce profitability.  
        \item \textbf{Land competition:} Limited area prioritised for cash crops reduces space for legumes.  
        \item \textbf{Knowledge and breeding gaps:} Few improved varieties and limited agronomic extension services.  
        \item \textbf{Agronomic complexity:} Some species are perennial, complicating short-cycle production and synchronisation with other crops.  
    \end{itemize}

    \textbf{Conclusion:}  
    Integrating legumes-whether as green manure or cover crops-offers long-term gains in soil fertility and resilience but requires policies, market incentives, and labour-efficient management to enable smallholders to adopt them sustainably in tropical production systems.
\end{enumerate}
\newpage
\section{Question 19 - Tropical Crop Physiology}

\begin{enumerate}
    \item \textbf{What are the four major environmental factors influencing crop evapotranspiration (ET), and how do they affect it?}

    Evapotranspiration (ET) combines soil evaporation and plant transpiration, reflecting the energy and water exchange between the surface and the atmosphere. It is driven by climatic and physiological conditions that determine water demand and crop growth rate.

    The four major environmental factors are:
    \begin{itemize}
        \item \textbf{Temperature:} Regulates the vapour pressure deficit (VPD); higher temperatures increase atmospheric demand for water vapour, accelerating ET and plant water use, whereas low temperatures reduce both ET and growth.  
        \item \textbf{Water availability:} ET depends on soil moisture; when rainfall is below crop water requirements, transpiration and photosynthesis decline. Rapid germination enables highland crops to exploit available soil water before drought periods.  
        \item \textbf{Solar radiation:} Supplies the latent heat for evaporation and drives photosynthesis and flowering, strongly influencing daily ET rates.  
        \item \textbf{Air movement (wind):} Replaces humid boundary layers around leaves, maintaining the vapour gradient and enhancing ET. In calm conditions, limited air exchange reduces ET.  
    \end{itemize}

    These factors interact dynamically-temperature and radiation create the evaporative demand, while soil water and canopy structure determine the supply. Managing this balance is crucial for productivity in tropical highland systems.

    \vspace{0.5em}
    \item \textbf{Why is stomatal conductance important in controlling crop transpiration, and how is it regulated under drought stress?}

    Stomatal conductance determines how open the stomata are, regulating both transpiration and CO$_2$ uptake. By adjusting stomatal aperture, crops maintain water balance while sustaining photosynthesis. Efficient water users such as \textit{ahipa} show high water-use efficiency through tight stomatal control.

    Under drought stress, regulation involves several adaptive strategies:
    \begin{itemize}
        \item \textbf{Reduced leaf area:} Species like cassava shed older leaves to lower the transpiring surface.  
        \item \textbf{Early water use:} Crops such as \textit{cañahua} germinate and establish rapidly to use available soil moisture before the dry period.  
        \item \textbf{Intrinsic drought tolerance:} Crops like \textit{mauka} and \textit{cañahua} maintain low transpiration rates or resilient stomatal function, sustaining minimal gas exchange under low water potential.  
    \end{itemize}

    Together, these mechanisms optimise the trade-off between carbon gain and water conservation, ensuring survival and stable yields in water-limited tropical environments.
\end{enumerate}

\newpage
\section{Question 20 - Rice}

\begin{enumerate}
    \item \textbf{What is the difference between upland and lowland rice production systems?}

    \textbf{Lowland (paddy) rice:}  
    Cultivated in flooded or semi-flooded fields where water levels are actively managed using bunds or ridges. The soil remains saturated for most of the crop cycle, suppressing weeds and promoting anaerobic microbial activity that enhances nutrient availability (e.g. Fe, Mn reduction). Flooding also stabilises temperature and water supply but leads to methane emissions. Lowland systems typically show higher and more stable yields due to controlled water management.

    \textbf{Upland rice:}  
    Grown under rainfed, non-flooded conditions in dryland fields. It depends entirely on rainfall, requiring well-drained soils and drought-tolerant varieties. The aerobic soil environment favours nitrification but increases susceptibility to nutrient leaching and weed pressure. Yields are generally lower and more variable due to drought stress and limited nutrient retention.

    \vspace{0.5em}
    \item \textbf{Provide examples of countries where each of these production systems can be found.}

    \begin{itemize}
        \item \textbf{Lowland (paddy) rice:} Common in China (e.g. Sichuan and Chengdu provinces), where farmers also cultivate \textit{Pachyrhizus} (yam bean) or soybean on ridges between paddies to improve soil fertility and diversification.  
        \item \textbf{Upland rice:} Typical of Central American dryland systems, where crops rely entirely on rainfall and are often intercropped with legumes or root crops, similar to yam bean-based production systems.  
    \end{itemize}

    \vspace{0.5em}
    In summary, lowland systems ensure high productivity and weed control through water management, while upland systems prioritise resilience and low-input adaptability under rainfed tropical conditions.
\end{enumerate}

\newpage
\section{Question 21 - Legumes as Nutrient Providers}

\begin{enumerate}
    \item \textbf{How can legumes play an important role in tropical production systems? Detail the mechanism.}

    Legumes enhance soil fertility in tropical systems primarily through Biological Nitrogen Fixation (BNF). Symbiotic \textit{Rhizobium} or \textit{Bradyrhizobium} bacteria within root nodules convert atmospheric N$_2$ into ammonia (NH$_3$), which the plant assimilates for protein and biomass formation. This process can fully supply the nitrogen demand of crops such as \textit{Pachyrhizus ahipa} without external fertiliser inputs. When legume residues are incorporated after harvest, a substantial portion of the fixed N (up to 215\,kg\,N\,ha$^{-1}$) remains in the soil, improving fertility for subsequent crops. Beyond nitrogen supply, legumes lower input costs, support agroecological sustainability, and diversify production through intercropping and mixed systems, enhancing resilience to climatic variability.

    \vspace{0.5em}
    \item \textbf{Are legumes always an advantage for the following crop in the crop rotation? How to assess if a legume is advantageous for the following crop?}

    Legumes are generally beneficial due to their positive nitrogen balance, but continuous cultivation can lead to pest, disease, or nematode buildup. For instance, \textit{Pachyrhizus erosus} performs poorly when grown repeatedly and requires 3–4 years before replanting to restore soil and pest equilibrium.

    To assess whether a legume benefits the following crop:
    \begin{itemize}
        \item Measure the \textbf{soil nitrogen balance} or total fixed N.  
        \item Compare the \textbf{yield response} of the subsequent crop against a non-legume rotation.  
        \item Evaluate \textbf{nodule formation and efficiency} through rhizobial inoculation and plant biomass.  
    \end{itemize}
    A measurable improvement in soil fertility or yield confirms the legume’s advantage within the rotation.

    \vspace{0.5em}
    \item \textbf{How do nitrogen fertilisers interact with the ability of a legume to fix nitrogen?}

    External nitrogen fertilisation suppresses BNF because legumes preferentially absorb available mineral N rather than forming new nodules. As a result, high fertiliser levels reduce symbiotic activity and biological efficiency. Efficient fixers such as \textit{P. ahipa} meet their N requirements entirely through BNF, making mineral N applications unnecessary and economically inefficient. Maintaining low external N encourages effective nodulation and maximises the ecological benefit of legumes in tropical production systems.
\end{enumerate}

\newpage
\section{Question 22 - Minor Cereals}

\begin{enumerate}
    \item \textbf{Give examples of minor cereals with local importance in the tropics.}

    In tropical highlands, locally important minor cereals include \textit{cañahua} (\textit{Chenopodium pallidicaule}) and \textit{quinoa} (\textit{Chenopodium quinoa}), traditional Andean grains cultivated between 3800–4200\,m\,a.s.l. for their exceptional protein content and micronutrient richness. Among true cereals (Poaceae), \textit{maize} (\textit{Zea mays}) and \textit{wheat} (\textit{Triticum aestivum}) also hold regional significance as short-cycle, market-oriented crops that increasingly replace traditional polycultures.

    \vspace{0.5em}
    \item \textbf{How is the yield of sorghum and millet in the tropics compared to other cereals?}

    Although precise yield data are not provided, minor cereals such as \textit{cañahua} and \textit{quinoa} typically yield less than major cereals under optimal conditions but maintain higher yield stability under stress. Their capacity to germinate and mature rapidly in drought, frost, or heat conditions makes them more reliable in marginal environments, ensuring food security where sorghum, millet, or maize yields would fluctuate strongly. This adaptability compensates for their lower productivity through consistent harvests in challenging tropical and high-altitude ecosystems.

    \vspace{0.5em}
    \item \textbf{As a consequence, how are they commonly grown in the tropics (what type of cropping system)?}

    Minor cereals are mainly cultivated in low-input, diversified systems adapted to local environments:
    \begin{itemize}
        \item \textbf{Agropastoral systems:} Integration of quinoa and cañahua with llama husbandry supports nutrient recycling through manure and provides livelihood diversity.  
        \item \textbf{Rotational systems:} Alternated with potatoes or legumes to exploit residual fertility and maintain soil health.  
    \end{itemize}
    However, in expanding commercial zones, these crops are increasingly grown as monocultures to meet export demand, reducing agrobiodiversity and competing with traditional grazing areas.

    \vspace{0.5em}
    \item \textbf{What are the two main factors determining the choice of cropping system?}

    \begin{enumerate}
        \item \textbf{Environmental adaptation and resilience:} Selection of crops and systems suited to cold, drought, poor soils, and short growing seasons-key survival traits in tropical highlands.  
        \item \textbf{Socio-economic viability and labour demand:} Farmers favour short-cycle, marketable crops requiring less manual labour and ensuring economic return over traditional labour-intensive systems.  
    \end{enumerate}

    \vspace{0.5em}
    In summary, minor cereals remain essential for sustainable tropical farming due to their ecological resilience and cultural value, even as socio-economic pressures drive transitions toward more uniform, market-driven systems.
\end{enumerate}