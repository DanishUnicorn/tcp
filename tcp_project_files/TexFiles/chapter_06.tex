\chapter{Exam Questions and Answers}
\setlength{\headheight}{12.71342pt}
\addtolength{\topmargin}{-0.71342pt}

This chapter of the course notes compiles the exam questions for the course held in November 2025, along with their respective answers prepared by me. The purpose of this section is twofold: firstly, to provide a reflective exercise that consolidates understanding of the course material; and secondly, to document my comprehension of the course topics as assessed through the exam questions.

\vspace{1em}
To ensure citation accuracy and academic transparency, NotebookLM has been employed as the primary generative AI platform. Its use has focused on verifying that all citations accurately reference the uploaded course materials and lecture slides provided by the professors. Beyond citation control, this section also represents an ongoing exploration of prompt engineering - refining interaction design to optimise AI output quality, precision, and academic reliability. Through this approach, the work aims to maintain a high academic standard while enhancing clarity, structure, and depth in written responses.

\vspace{1em}
There are a total of 17 questions in the exam, each comprising between three and five sub-questions. The numbering of the sections in this chapter corresponds directly to the numbering of the exam questions, ensuring a clear and consistent structure throughout. Questions 1-9 address aspects related to crop physiology, while questions 10-17 focus on fruit quality, maturity, and usability. Each question is presented below, followed by its respective sub-questions and answers.

\newpage
\section{Question 1 - Highland Crops}
\textbf{How would you make the association between crop calendar and climate change. Specify links between crop phenology and climate knowledge. You can use an example:}

Crop phenology describes the timing of developmental stages controlled mainly by temperature and moisture, forming the scientific basis of traditional crop calendars. Farmers align sowing and harvesting with favourable climatic windows to optimise germination, growth, and yield. Phenological events follow the principle of thermal time — the accumulation of degree days above a base temperature. For example, cañahua germination begins at about -0.9\textdegree C and requires 476\textdegree C hours, illustrating adaptation to cold, highland environments.

\vspace{0.5em}
In regions like the Bolivian Altiplano, the crop calendar is fixed by short rainy seasons and low temperatures; early sowing between September and November allows crops to establish before drought. Climate shifts disturb this synchrony by altering temperature and rainfall patterns, shortening or lengthening growth cycles. In ahipa, later sowing reduced yield due to a shortened growth period, and comparisons between the 1990s and 2010s show reduced cycle durations, reflecting environmental change. Farmers increasingly replace long-cycle crops with shorter, marketable ones to manage risk.

\vspace{0.5em}
Genetic diversity within native crops, such as frost-tolerant Solanum species, provides resilience, allowing flexible crop calendars under changing climates. Thus, the crop calendar represents an integration of phenological knowledge and climate adaptation, where thermal time and temperature thresholds govern planting decisions and responses to climate change.

%Links between Crop Phenology and Climate Knowledge
%Crop phenology, which describes the timing of developmental stages (e.g., germination, flowering, maturity), is closely controlled by climate, especially temperature. Farmers' traditional knowledge about crop calendars and cultivation practices are inherently linked to coping with existing climate constraints and maximizing favorable weather conditions.
%1. Thermal Time as a Phenological Predictor
%A central link between phenology and climate knowledge is the principle of thermal time.
%• The progress of seed germination and seedling growth are closely related to the "thermal time". When moisture and oxygen are available, germination and seedling growth are controlled by the number of accumulated degree days above a certain base temperature (T$_b$).
%• This physiological relationship allows for the estimation of the thermal time required for specific phenological events. For example, in the cañahua cultivar Warikunca, the calculated base temperature for 50% germination was $-0.9\text{ ^\circ}\text{C}$, and the thermal time required for germination was 476 $\text{^\circ}\text{C}$ hours. Low base temperatures characterize native cultivars from cold and dry environments.
%• The determination of optimum temperatures for growth and development through physiological studies is required to establish the limits for crop expansion. Furthermore, low temperature is a main factor to be considered as it reduced productivity in areas of cool nights.
%2. Climate Constraints Defining the Crop Calendar (Sowing Date)
%The agricultural calendar (sowing and harvesting dates) in adverse environments like the Bolivian Altiplano is fixed by challenging climate conditions, which directly influence phenological requirements.
%• The climate of the Bolivian Altiplano is cold with a short growing season. Andean crops, such as cañahua, are typically sown between September and November. This early sowing is vital because rapid germination at low temperatures is important to obtain early crop establishment and exploit soil water before dryer periods set in.
%• Andean farmers traditionally use local landraces, recognizing that the rainy season is short and that reduced soil water availability and low temperatures are common during land preparation and planting.
%• The source documents reveal that changes in sowing dates significantly affect phenology and yield. For the root crop ahipa, delaying the sowing date did not affect flowering time but negatively affected root and seed yield by shortening the growth cycle. Ahipa root yield correlated with a gross estimate of heat-units based on the minimum temperatures accumulated during the growth season.
%• Climate indicators also define the end of the crop calendar: physiological maturity in cañahua is indicated by a color change of the plant. This timing depends on the climate; if the days become cold, this stage will have a shorter duration.
%Association between Crop Calendar and Climate Change
%Climate change manifests through shifting temperature and rainfall patterns, directly impacting when farmers can plant (crop calendar) and how long crops take to mature (phenology), thus threatening traditional agricultural systems.
%1. Shifts in Growth Cycle Duration
%The comparison of cropping periods (e.g., 1994/96 versus 2012 for ahipa) illustrates that the duration of the ahipa growth cycle differed significantly among landraces between the periods studied. For example, landraces in the La Paz River basin had a growth period of 211 days in 1994/96, but only 166 days in 2012. Although the sources relate these changes mostly to socio-economic drivers, such shifts in phenological traits over time reflect adaptations or vulnerabilities to changing environmental conditions (which may include climate change effects).
%2. Marketable Crops Replacing Traditional Crops with Long Phenological Cycles
%A clear consequence related to climate variability and economic pressures is the shift away from traditional crops (orphan crops) that often have long growth cycles, in favor of marketable crops with shorter growth cycles.
%• The growth cycle of ahipa is relatively long (between 6 and 8 months). Priority is increasingly given to short growth-cycle cash crops, such as cereals (Poaceae) and vegetables (Solanaceae), which are less labor-demanding.
%• This preference for shorter-cycle crops suggests that farmers are managing risk, possibly due to increasing climate unpredictability (e.g., changes in the timing or reliability of the rainy season), which makes long-duration crops more vulnerable.
%3. Resilience through Genetic Diversity
%The association between the crop calendar and climate knowledge is also evident in the inherent resilience of specific landraces, which farmers maintain as an adaptation strategy.
%• In the context of climate change, the diversity within native potatoes is crucial. Some native potato varieties, such as S. juzepczukii, S. ajanhuiri, and S. curtilobum, are highly tolerant to harsh climatic conditions like frost and drought.
%• Maintaining cropping combinations using different species and varieties can reduce the exposure of the overall crop system to climate stressors.
%In summary, the sources indicate that farmers' crop calendars are tightly regulated by regional climate (temperature and rainfall patterns). Phenology provides the mechanistic link, as key growth phases like germination are governed by accumulated thermal time. Climate change threats are reflected in the observed shifts in crop cycle lengths over time and the adaptive strategies involving switching to shorter-cycle, marketable crops to minimize risk.
%The sources best suited to explain this relationship are:
%• The cañahua germination study which introduces thermal time, base temperature, and early establishment in cold climates.
%• The ahipa agronomy study which highlights how sowing date and minimum temperatures affect phenological cycle length and overall yield.
%• The native potato discussion which directly links tolerance to frost and drought to crop resilience in a climate change context.


\section{Question 2 - Highland Crops}
\textbf{Which are the social constrains when introducing new varieties? And other factors of agronomic importance? }

Introducing new highland crop varieties faces strong social barriers linked to market, cultural, and knowledge-related factors. Economically, farmers prefer short-cycle, high-value crops like cereals and vegetables because traditional species such as ahipa or mauka lack market demand, stable prices, and logistical access to urban centres. Culturally, native roots and tubers are often stigmatized as “poor man's food” and rejected for their pungent or astringent flavours. Urban migration has further shifted diets toward processed staples, weakening the link between consumers and traditional crops. The loss of traditional agricultural knowledge, as younger generations leave rural areas, also limits the capacity to manage, process, and appreciate these species.

\vspace{0.5em}
Agronomically, highland crops show remarkable adaptation to extreme conditions such as frost, drought, and poor soils. Species like cañahua and native Solanum potatoes thrive at high altitudes and under minimal input conditions. However, productivity remains limited by issues like seed shattering in cañahua, virus incidence in ulluco and mashua, and competition between roots and reproductive shoots in ahipa. Practices such as pruning, seed inoculation with rhizobia, and improved soil fertility management could substantially raise yields. Yet, these crops often receive fewer resources compared to commercial staples, reflecting the combined influence of social constraints and agronomic challenges on their wider adoption.


%1. Social Constraints When Introducing New Varieties
%The key social constraints that impede the introduction or widespread adoption of new (or traditionally conserved) highland crop varieties largely stem from market dynamics, cultural stigma, and the loss of traditional agricultural expertise.
%A. Market and Economic Factors
%• Competition from Marketable Crops: The conservation of on-farm agro-biodiversity is threatened by monoculture and marketable crops. Farmers prioritize short growth-cycle cash crops that are less labor demanding, such as cereals (Poaceae) and vegetables (Solanaceae), including white maize cobs, tomato, and potato.
%• Low Profitability and Lack of Market Access: The price of some traditional crops, such as ahipa, had not increased in 70% of the urban markets between 1994/96 and 2012, offering little incentive to retain their cultivation. This issue is compounded by the historical lack of link between peasants and markets.
%• Remoteness and Logistics: Remoteness of communities from urban centers is a major factor impacting population dynamics and the commercialization of crops. The transportation of agricultural products to the nearest bus station is often very cumbersome.
%B. Cultural Stigma and Consumer Habits
%• Stigma as "Poor Man's Food": Many Andean roots and tubers suffered from the prejudice of being regarded as ‘food of the poor' following the Spanish conquest, which classified them as inferior to European crops like barley and wheat. This social stigmatization persists, discouraging consumption in urban areas.
%• Changing Diets: Migration to urban centers has led to the abandonment of traditional food sources due to the intake of cheaper, processed foods and a shift towards basic foods like pasta, rice, and bread.
%• Sensory/Quality Issues: Some native crops have traits that are socially limiting. For example, the tubers of mashua have a pungent flavor due to their isothiocyanate content, which causes scarce popularity. The pungency of mauka (astringent taste) reported in some landraces must be addressed, as it does not attract new consumers.
%C. Loss of Traditional Knowledge
%• Aging Farmer Population: The younger generations have migrated to the cities, leading to a situation where most traditional growers (e.g., ahipa farmers) are older, often more than 50 years old. This situation accelerates the loss of traditional knowledge and experience related to crop husbandry, cultivation, seed management, and post-harvest treatments (like sun exposure to sweeten tubers).
%• Loss of Specific Techniques: Traditional techniques that enhance the usefulness and enjoyability of crops, such as aporque (hilling), sweetening mauka roots in the sun, and using foliage as livestock feed, are becoming unknown to a significant proportion of farmers, which further reduces local interest in the crop.

%--------------------------------------------------------------------------------
%2. Other Factors of Agronomic Importance
%The sources highlight several critical agronomic factors concerning highland crops, spanning environmental constraints, management practices, and yield efficiency challenges.
%A. Adaptation to Extreme Environmental Conditions
%• Climate Resilience: Many highland crops exhibit remarkable resilience, which makes them sustainable in adverse environments.
%    ◦ Cañahua (Chenopodium pallidicaule): Landraces tolerate low temperatures down to 0 $\text{^\circ}\text{C}$, drought, and saline soils.
%    ◦ Native Potatoes (Solanum spp.): Cultivation systems allow these crops to withstand frost and drought.
%    ◦ Mashua and Oca are known to be frost-tolerant and withstand intense cold.
%• Altitude and Soil: Crops like cañahua are grown in the Altiplano between 3810 and 4200 m a.s.l.. Ulluco and mashua are commonly found between 3000 and 3800 m a.s.l.. Mashua requires loose soils with a slightly acidic pH (5 to 6), though it develops at a pH up to 7.5.
%B. Management and Input Requirements
%• Low Input Needs: Many underutilized species, such as ahipa and mauka, are rustic and do not require pest and disease control.
%• Biological Nitrogen Fixation: Ahipa is a N2 -fixing legume, and when in symbiosis with rhizobia, it can be grown without N fertilizers. Seed inoculation with effective rhizobia greatly increased root and seed production in ahipa.
%• Reproductive Pruning: For root crops like ahipa, the manual removal of reproductive shoots (pruning) is generally practiced to reduce fruit-root competition for carbon assimilates and nutrients. Pruning can dramatically increase root yield (e.g., two to five times for some accessions of ahipa). However, this practice requires intensive manpower.
%• Fertilization: Mauka, a marginal crop, is usually not fertilized because available livestock manure is primarily applied to the main cash crops, such as potatoes and maize.
%C. Yield and Productivity Constraints
%• Seed Shattering: High seed shattering rates remain a major problem in the production of cañahua, with losses under field conditions ranging from 15-21% during flowering to 25-35% during physiological maturity. This trait is highly bred out in domesticated plants.
%• Virus Incidence: High virus incidence limits the productivity of ulluco and mashua. The sources suggest that cleaning ulluco from viruses could increase its production dramatically.
%• Root Yield Potential: While high experimental yields are possible (e.g., mashua up to 70-80 t ha-1; mauka 30 to 58 t ha-1 typical), commercial yields are often much lower (e.g., mashua 5-15 t ha-1). Mauka's potential as a productive crop is demonstrated by its harvest index (49% to 51% dry matter allocated to roots), which is higher than yacón and close to improved potato varieties.
%• Soil Fertility and Yield Gap: High plant density resulted in low root yield for some ahipa landraces, which may be a result of low soil fertility. A significant fraction of the yield gap could be avoided by proper agronomic management.
%• Establishment: For cañahua, a fast and uniform establishment of plants in the field is cited as one of the most important factors for efficient crop production. The traditional method of scattering seeds and lightly covering them (shallow seeding) is practiced.

\newpage
\section{Question 3 - Intercropping}
\textbf{Explain the concept of intercropping and provide potential benefits of practicing intercropping in agriculture.}

Intercropping is the cultivation of two or more crop species within the same plot, a practice common in traditional Andean farming systems. It involves combinations such as ahipa with tomato, onion, or maize; mauka with maize and beans; or mixed tuber systems of oca, ulluco, and mashua. These systems integrate food, fodder, and cash crops within a single field, optimising limited land resources.

\vspace{0.5em}
The benefits of intercropping include enhanced soil fertility through biological nitrogen fixation by legumes like ahipa, which reduces the need for synthetic fertilizers. It improves pest and disease resilience, as mixed species can protect one another and reduce pathogen spread — for instance, mauka intercropped with maize shows greater frost tolerance and health due to antimicrobial compounds in its roots. Intercropping also increases yield stability and resource efficiency by producing multiple outputs (food and fodder) from the same area.

\vspace{0.5em}
Moreover, intercropping sustains agrobiodiversity and maintains traditional polyculture systems that balance ecological, economic, and cultural resilience. This diversity strengthens environmental stability and supports long-term agricultural sustainability in highland systems.

%1. Explaining the Concept of Intercropping
%The sources describe intercropping as the practice of growing multiple crops together within the same plot or farming system. This is associated with traditional farming systems or polyculture systems used by smallholder farmers in areas like the Inter-Andean Valleys (IAV) and the Bolivian Altiplano.
%Key examples from the sources illustrating the concept include:
%• Ahipa and Other Crops: Farmers cultivate ahipa (Pachyrhizus ahipa) in traditional polyculture systems. These plots may include tomato (Solanum lycopersicum L.), onion (Allium cepa L.), fennel (Foeniculum vulgare Mill.), arracacha, groundnut (Arachis hypogaea L.), cassava (Manihot esculenta Crantz), white giant maize corn (Zea mays L. var. cuzcoensis Kornicke), and achoccha (Cyclanthera pedata Schrad.).
%• Mauka and Maize: Mauka (Mirabilis expansa) is frequently intercropped with maize. It is also found cultivated in assortment with maize, beans (Phaseolus vulgaris L.), yacón, and squash (Cucurbita pepo L.). Sometimes, mauka grows spontaneously in wheat crops.
%• Arracacha in Mixed Systems: Arracacha is either monocropped or in association with maize, or beans (Phaseolus spp.), or intercropped with other crops in the region. In the community of Lloja (Bolivia), arracacha plants were observed intercropped with cassava.
%• Cassava in Polyculture: Cassava is typically cultivated in a polyculture system with other floodplain crops.
%• General Andean Practice: The tuber-bearing plants oca, ulluco, and mashua are traditionally planted together either mixed or in adjacent plots.
%• Achira in Familiar Cropping Systems: Smallholders practice intercropping and polyculture involving achira (Canna indica L.).
%2. Potential Benefits of Practicing Intercropping
%The sources associate intercropping and mixed cropping with several significant benefits, primarily related to agricultural sustainability, resource use efficiency, and risk reduction.
%A. Sustainable Land Use and Soil Health
%• Biological Nitrogen Fixation: Intercropping legumes, specifically the Pachyrhizus species, with other crops promotes soil fertility. As an N2 -fixing legume, ahipa does not require nitrogen fertilization. The genus Pachyrhizus has an efficient symbiosis with nitrogen-fixing Rhizobium and Bradyrhizobium bacteria, providing plants with nitrogen. This feature allows the crop to form an integral part of a sustainable land-use system, from both an ecological and a socioeconomic standpoint.
%• Crop Rotation: Ahipa is cited as an interesting crop to include in rotations. Farmers practice crop rotation, often shifting from a root or tuber crop with nematode problems to a grain crop, e.g., maize, the next year.
%B. Crop Protection and Resilience
%• Pest and Disease Management: The sources emphasize the inherent resilience to pests and diseases in these mixed systems. Mauka's aptitude for self-defense is noted, with some informants suggesting that intercropping maize may shelter mauka from frost damage. The fact that mauka is generally healthy and the ribosome-inactivating proteins synthesized from its storage roots have an antimicrobial effect against root-rotting microorganisms reinforces its suitability for mixed systems.
%• Yield Reliability: The combination of a tuberous root crop (like Pachyrhizus) with a legume offers the yield reliability of a tuberous root crop, combined with the high sustainability of a legume.
%C. Optimized Resource Use
%• Multiple Outputs (Fodder/Food): Intercropping systems allow farmers to maximize outputs from small plots of land. Achira leaves can be a good alternative for fodder in rural households. The dried hay of Pachyrhizus is used as fodder. The ability of mauka to be used for both leaves and roots also enhances resource utility in these mixed systems.
%• Adaptability: The cultivated Pachyrhizus groups easily adapt to small-farmers systems, as it can be intercropped with maize and bean. This versatility is crucial in the challenging Andean environment.
%D. Preservation of Agrobiodiversity
%• Maintaining Diversity: Intercropping is a foundational element of the traditional farming systems (e.g., chiru, ananta, and k'ata), which enable farmers to maintain a high agro-biodiversity. High agro-biodiversity cultivated on small land sizes is characteristic of these communities. The practice of growing many species together contributes to the conservation of the environment and biodiversity.


\section{Question 4 - Intercropping}
\textbf{Discuss the challenges and potential disadvantages of intercropping in modern agricultural systems. Provide examples of situations where intercropping may not be the best strategy.}

Intercropping, though ecologically beneficial, faces several challenges in modern agricultural systems. Agronomically, competition between species can reduce yield efficiency. Fast-growing crops like mauka can overshadow companions, and systems prioritising multiple outputs—such as leaf and starch in achira—often compromise the main yield. High plant density in mixed systems may also limit root yield, as seen in ahipa.

\vspace{0.5em}
Labour intensity is another major limitation. Practices like reproductive pruning in ahipa can multiply yields but demand extensive manpower, making such systems economically unsustainable compared to low-labour monocrops. Market and industrial pressures further discourage intercropping, as these systems produce non-uniform, small-scale outputs that do not meet commercial processing standards or large-volume demands.

\vspace{0.5em}
Intercropping is thus less suitable where uniformity, mechanisation, or rapid returns are required — for instance, industrial ahipa or achira production aiming for starch extraction, or in intensified zones where monocultures dominate for efficiency. Additionally, competitive crops like mauka are better placed along field borders rather than intercropped. Hence, while intercropping sustains biodiversity, its complexity and low profitability limit its adoption in modern, high-output agriculture.


%Challenges and Potential Disadvantages of Intercropping
%The sources highlight several critical constraints related to agronomy, labor, and economic pressure that challenge the sustainability and adoption of traditional intercropping systems in a modern agricultural context:
%1. Agronomic Constraints (Competition and Yield Inefficiency)
%A primary disadvantage of growing multiple species together is the potential for competition and reduced output efficiency compared to specialized cultivation:
%• Compromised Crop Growth: Intercropping can lead to significant competition, particularly when one crop is fast-growing and aggressive. The foliage of Mauka (Mirabilis expansa), due to its decumbent nature, grows and expands rapidly, thus compromising the growth potential other species when cultivated in association. Farmers reported that close intercropping compromised root production for mauka.
%• Yield Reduction in Specialized Production: In integrated systems that prioritize multiple outputs (like food and fodder), maximizing one output can negatively affect the primary yield component. For achira (Canna indica), the leaf mass production provides environmental benefits and fodder, but increasing leaf yield can reduce starch yield.
%• Negative Density Effects: While high diversity indices characterize traditional farming systems, high plant density (sometimes associated with mixed systems on small plots) resulted in low root yield for ahipa landraces in a study, potentially as a result of low soil fertility. Conversely, increasing plant density in ahipa had a negative effect on root and pod growth per plant.
%2. Labor and Management Demands
%Traditional intercropping systems often rely on specific, labor-intensive practices that are disadvantageous in modern systems where labor costs are high and alternative livelihoods exist:
%• Intensive Manpower Requirement: Traditional cropping of the root legume ahipa requires a laborious yield-enhancing practice: reproductive pruning (manual removal of flowers or reproductive shoots) to redirect assimilates to the tuberous root. Although pruning dramatically increases root yield (more than five times for some accessions), this operation requires intensive manpower.
%• High Production Costs: The complexity and labor demands of traditional polyculture are becoming economically unviable. Priority is increasingly given to short growth-cycle cash crops that are less labour demanding. Commercial crops are replacing ahipa cultivation because of their quick economic returns and because they require less care.
%• Cumulative Labor Demands: The shift toward modern agriculture is driven by the fact that traditional crops like ahipa require a lot of labor, making them less competitive than marketable crops that require less labor.
%3. Market and Modernization Pressure
%The inherent nature of intercropping often clashes with the demands of modern markets and industrialization, leading farmers to abandon these practices in favor of monoculture:
%• Threat of Monoculture: The conservation of on-farm agro-biodiversity, typically maintained through mixed cropping, is threatened by monoculture and marketable crops. Farmers are increasingly opting for monocrops and crops with short growth cycles.
%• Lack of Standardization: Traditional farming systems are based on small land sizes, and the resulting production volume and variability are often inconsistent for modern market demands, leading to a low market value for crops like ahipa.
%• Loss of Knowledge: The dynamics of change cause the traditional cropping systems to rapidly disappear, along with the associated ethno-ecological knowledge required to manage complex intercropping systems effectively.
%Examples of Situations Where Intercropping May Not Be the Best Strategy
%Intercropping is demonstrated to be suboptimal when the goals shift toward commercial efficiency, industrial production, or risk mitigation related to specific crop characteristics:
%1. Industrial Processing Requiring Large, Uniform Volumes:
%    ◦ For the industrial processing of ahipa roots, the high cost of manpower associated with manual reproductive pruning (necessary in many traditional systems) could be avoided by selecting low flowering landraces and/or increasing planting density. This implies that a specialized, dense monoculture would be necessary to avoid the labor constraint inherent in the polyculture approach if the objective is competitive industrial raw material production.
%    ◦ Similarly, for starch extraction from achira, if maximizing starch output is the goal, maximizing leaf yield for fodder (a benefit in intercropping) must be avoided, favoring systems focused solely on rhizome production.
%2. When Specific Competitive Crops are Involved:
%    ◦ Intercropping mauka with other species may not be the best strategy for maximizing the yield of the companion crop due to mauka's tendency for rapid foliage expansion which compromises the growth potential of others. Farmers sometimes plant Mauka along the edge of the main crops, rather than closely intercropped, mitigating this competitive disadvantage while retaining some level of integration.
%3. In Environments with Intensified Land Use and High Economic Priority:
%    ◦ In areas where intensified land use is adopted, such as the lower altitudes of the East Andean valley in Bolivia and Manabí, Ecuador, monocropping is the rule and this intensive approach does not allow the practice of shifting cultivation (a traditional form of rotational mixed cropping). Farmers choose this pathway because marketable crops provide quick economic returns, suggesting that intercropping is suboptimal when short-term profitability and land efficiency are paramount.
%4. When Managing Land-Use Conflicts (Agropastoral Systems):
%    ◦ In the Bolivian Altiplano, the expansion and intensification of quinoa cultivation for cash cropping created a direct competition for land use with traditional llama husbandry. While quinoa was traditionally cultivated alongside llama husbandry in a balanced rotation, the intensified, specialized focus on quinoa (monoculture trend) led to llamas grazing on quinoa plants and pastoralists being forced to pasture their animals in further marginalized areas. This demonstrates that specialized, high-intensity systems supplant mixed systems when economic incentives shift, creating a land-use conflict where the mixed system is no longer maintained.


\newpage
\section{Question 5 - Seed Germination}
\textbf{Germination test of three seed lots of cowpea supplied by a farmer }

\begin{figure}[h]
    \centering
    \includegraphics[width=0.35\textwidth]{Figures/tcp_q_05.JPG}
    \caption{The relation between percent germinated seed over time of three cowpea seed lots.}
    \label{fig:tcp_q_05}
\end{figure}

1) Describe the curves.
All three are cumulative germination-time curves. They differ in onset (time to first germination), rate (slope), and final fraction germinated. Curve a begins earlier and rises faster; b is delayed and slower but approaches a high final level; c shows a different shape with slower rise and a lower/asymptotic final fraction.

2) Are they as expected for an ordinary germination test?
Yes. Under ISTA-type conditions (controlled temperature/light; counts over time) cumulative germination typically follows a sigmoidal increase, with seed lots differing mainly in rate and time to a given percentile.

3) How to estimate the germination curve mathematically?
Fit the counts-over-time to the cumulative distribution function of the standard log-logistic model to obtain F(t), the fraction germinated at time t.

4) Necessary parameters in the model.
The model requires at least: F(t) (cumulative fraction), and t50 (time to 50\% germination). Differences in curve steepness are captured by the model's shape parameter (as reflected in the fitted curve).

5) How to test whether seed lots a and b differ significantly?
Fit the log-logistic model to each lot's time series and compare the fitted parameters—especially t50 (and shape)—between lots; statistical difference is inferred from non-overlap of fitted parameter estimates under the same ISTA test conditions.

6) Why does seed lot c have another shape than a and b?
It reflects different fitted parameters: a lower final fraction germinated and/or a different slope (shape) and t50, indicating slower and incomplete germination under the same controlled test.

7) If lots a and b are identical but a is from an ordinary germination test, what test could yield curve b?
A seedling emergence test (following the standard emergence protocol) rather than a germination test—i.e., different test conditions (substrate/temperature regime) that alter the time course and fitted parameters.


\begin{enumerate}
    \item \textbf{Describe the curves.}
        \begin{enumerate}
            \item this curve
            \item this curve
            \item this curve
        \end{enumerate}
    \item \textbf{The germination curves represent developments as expected for an ordinary germination test for the three seed lots?} 
    \item \textbf{How can you estimate the germination curve based on count over time using a mathematical model.}
    \item \textbf{Which parameters are necessary in the mathematical model?}
    \item \textbf{How can you test whether the curves for seed lots a and b are statistically significantly different.}
    \item \textbf{What could be the reason why the curve for seed lot c has another shape than seed lots a and b?}
    \item \textbf{If seed lots a and b are actually identical, but curve a is a result of an ordinary germination test, what kind of test could then result in curve b.}
\end{enumerate}

%Intercropping is characterized as the practice of cultivating multiple crops together within the same plot, frequently associated with traditional farming systems or polyculture systems utilized by indigenous smallholder farmers in regions such as the Inter-Andean Valleys (IAV) and the Bolivian Altiplano. These traditional farming systems, such as chiru, ananta, and k'ata, are dynamic units that have historically enabled farmers to maintain a high agro-biodiversity. Examples include the cultivation of the root legume ahipa (Pachyrhizus ahipa) in polyculture systems alongside crops such as tomato, onion, fennel, arracacha, groundnut, cassava, white giant maize corn, and achoccha. Likewise, mauka (Mirabilis expansa) is frequently intercropped with maize, beans, yacón, and squash, and other Andean crops like oca, ulluco, and mashua are traditionally planted together.
%The practice of intercropping provides several potential benefits, primarily centered on sustainability, resource optimization, and resilience. One major benefit is the integral role of legumes like ahipa in promoting a sustainable land-use system by acting as N2 -fixing crops, which allows them to be grown without N fertilizers. This feature, along with pest tolerance, reduces input requirements and lowers environmental impact and production costs. Furthermore, the combination of a tuberous root crop and a legume can offer the yield reliability of a tuberous root crop alongside the high sustainability of a legume. These systems allow farmers to obtain multiple outputs; for instance, dried hay from Pachyrhizus is used as fodder, provided reproductive pruning is implemented to avoid toxic rotenone. Certain crops within these agro-ecological systems also provide protection, as seen with mauka, which synthesizes ribosome-inactivating proteins that have an antimicrobial effect against root-rotting microorganisms and generally does not succumb to pests or diseases.
%Despite these benefits, intercropping systems face significant challenges and potential disadvantages in modern agricultural contexts, often leading to their decline in favor of marketable crops. The foremost constraint relates to labor and economic viability; traditional methods require intensive manpower. For ahipa, the laborious yield-enhancing practice of reproductive pruning (manual removal of flowers) is needed to increase root yield. Conversely, farmers are increasingly giving priority to short growth-cycle cash crops that are less labour demanding, such as cereals (Poaceae) and vegetables (Solanaceae). This low market value endangers conservation of crops like ahipa, which is associated with traditional farming systems involving ethno-ecological knowledge.
%Intercropping may also be a suboptimal strategy when facing certain agronomic challenges or when aiming for commercial efficiency. Competition between associated crops is a recognized issue. For example, the foliage of mauka, due to its decumbent nature, grows and expands rapidly, thus compromising the growth potential other species when cultivated in association. Consequently, farmers have reported that close intercropping compromised root production. In situations where industrial utilization is prioritized, monoculture may be more advantageous; for instance, the high cost of manual pruning for ahipa could be avoided by breeding low flowering landraces and/or increasing planting density in specialized production systems. The viability of intercropping is also severely challenged by land-use conflict; the expansion of quinoa production for cash cropping, often displacing traditional mixed fields, directly competes with and endangers the robust agropastoral ecosystems of the Bolivian Altiplano. Site 1 in one study reported that the most common limitation to ensuring sufficient pasture land was that the land was already in use for quinoa production, leading to a significantly lower perception of sufficient pasture land available.


\section{Question 6 - Sugar Production}
\begin{enumerate}
    \item \textbf{What is sugar?}  
    \item Sugar refers to simple carbohydrates such as sucrose, glucose, and fructose that serve as major energy sources. In many root and tuber crops, sweetness results from the conversion of complex carbohydrates or fructooligosaccharides (FOS) into these simple sugars through enzymatic hydrolysis, often enhanced by sun exposure.
    \item \textbf{Mention all the crops you known which are used for sugar production.}  
    \item Crops mentioned include yacon, oca, mashua, ahipa, achira, and sugarcane. These contain varying proportions of sucrose, glucose, and fructose. Yacon and oca are used for syrup or sweetener production, and sugarcane is used for molasses and syrup.
    \item \textbf{Sugar cane can be used for several purposes. Mention these.}  
    \item sugarcane is used to produce molasses (miel de caña), concentrated syrup, and sweeteners like chancaca. It can be processed for direct consumption, combined with other crops in traditional foods, or used as livestock feed when mixed with sweetpotato vines or foliage.
    \item \textbf{Explain how sugar cane normally is established in the field.} 
    \item No information is available in the sources on sugarcane establishment methods.
    \item \textbf{Is it necessary to establish a new sugar cane crop each year?}  
    \item No relevant information is provided in the sources regarding re-establishment frequency.
    \item \textbf{Why and when do farmers many places in the world ignite sugar cane fields?} 
    \item The sources provide no information on field burning practices for sugarcane
    \item \textbf{After sugar canes have been processed in a factory, which types of waste products are produced and what can they be used for?} 
    \item he sources contain no information on industrial sugarcane by-products or their uses.
\end{enumerate}

\newpage
\section{Question 7 - Cropping Systems}

A small subsistence farm is placed in a semi-arid area in the highlands of Guatemala (500 mm of annual precipitation). Crops are grown in rotation with a short fallow period, and fertilizer is not used on the subsistence crops. The field is not irrigated. Answer the following questions for a field grown with maize intercropped with bean. 

\vspace{0.5em}
Indicate the approximate onset and duration of the rainy season, give sowing and harvesting time for the two crops, and estimate a realistic yield for maize monocrop. Do you think the yields can improve with the intercropping, why?. 

\vspace{0.5em}
Suggest crop establishment for the two crops. What could be the advantages of intercropping the two species? 

\vspace{1em}
Rainy season, sowing, and harvesting
In semi-arid highlands with about 500 mm annual rainfall, the rainy season typically starts around September-November and lasts until March-April. Both maize and bean are sown at the onset of the rains (September-November). Beans mature earlier and are harvested around 60 days after sowing, while maize is harvested 110-150 days after sowing, typically between January and March.

Estimated yield for maize monocrop
A realistic yield for a rainfed, unfertilized maize monocrop under these conditions is about 1 t ha$^{-1}$.

Yield improvement with intercropping
Yields can improve because beans, as legumes, fix atmospheric nitrogen through symbiosis with Rhizobium bacteria, enriching soil fertility and benefiting maize. Intercropping also enhances water and nutrient use efficiency and reduces the risk of total crop failure under semi-arid conditions.

Crop establishment
The field should be ploughed using oxen or simple tools to retain moisture. Maize is planted in rows at the start of the rains, with beans sown between maize rows to maximize space and soil cover.

Advantages of intercropping maize and bean

Nitrogen fixation by beans improves maize nutrition and system sustainability.

Efficient land use and higher total productivity from the same area.

Risk reduction against drought and crop failure.

Structural benefits, as maize provides support and partial shade for beans.

Maintenance of agrobiodiversity, characteristic of traditional milpa-type systems.

\newpage
\section{Question 8 - Cropping Systems}
\textbf{Quinoa in Bolivia:}

\vspace{0.5em}
A farmer close to Titicaca Lake in Bolivia grows quinoa as one of his main crops. His village receives about 800 mm rain per year. 

\vspace{0.5em}
\textbf{How is rainfall and temperature distributed over the year? }
Rainfall and temperature distribution
In the highlands near Lake Titicaca, annual rainfall is about 800 mm, concentrated between September and April, corresponding to the short rainy and growing season. The dry season extends from May to November. Temperatures fluctuate strongly, with daytime maxima of 17-19\textdegree C and night minima near 0-3\textdegree C, occasionally dropping below freezing. The mean temperature during the cropping season is about 9-10\textdegree C, which suits quinoa's tolerance to cold conditions and low base temperature (around 3\textdegree C).

\vspace{0.5em}
\textbf{When is quinoa sown and harvested? }
Quinoa is typically sown from September to November, at the onset of the rains, to ensure good establishment. The crop matures with the decline of the rainy season and is harvested in April or May.

\vspace{0.5em}
\textbf{Which other crops may the farmer grow? }
Farmers near Lake Titicaca often combine quinoa with other traditional Andean crops and livestock. Common companion or rotation crops include potatoes (Solanum spp.), cañahua (Chenopodium pallidicaule), and other Andean root and tuber crops such as oca, mashua, and ulluco, all suited to 700-1000 mm rainfall and high altitudes. The farmer likely integrates llama husbandry into this agropastoral system to maintain soil fertility and livelihood diversity.

\newpage
\section{Question 9 - Intercropping}

1) Monocropping vs. Intercropping
Intercropping (polyculture) involves cultivating multiple species together, whereas monocropping focuses on one crop per field.

Advantages of intercropping:
Enhances agrobiodiversity and resilience, improving adaptation to climatic stress.
Promotes soil fertility through biological nitrogen fixation by legumes (e.g. ahipa).
Reduces production risk, as crop diversity buffers against total failure.

Disadvantages of intercropping:
Requires high labour input (e.g. pruning ahipa for yield).
Causes competition between crops, possibly lowering individual yields.
Has low marketability and scalability, unsuitable for mechanised or industrial systems.

Advantages of monocropping:
Provides high yield potential for specific commercial crops.
Offers quick economic returns and efficient management.
Facilitates mechanisation and easier pest and nutrient control.

Disadvantages of monocropping:
Causes loss of agrobiodiversity and increased vulnerability to pests.
Leads to environmental unsustainability and soil depletion.
Creates land-use conflicts, such as between quinoa fields and llama pastures.

2) Planning a trial to test intercropping effects
Use a randomised block or split-plot design with treatments comparing:

Monocrop A (e.g. maize)

Monocrop B (e.g. bean)

Intercrop A+B (alternate rows or mixed plots)

Include several replicates, equal plant densities, and similar soil conditions.
Record yield and yield components for each crop. Analyse data with ANOVA and compare means using Tukey's test. For multi-environment studies, use AMMI analysis to assess genotype $\times$ environment interactions.

3) Land Equivalent Ratio (LER)
LER measures how efficiently land is used under intercropping compared with monocropping: LER = (Yab/Ya) + (Yba/Yb) where Yab and Yba are intercrop yields of each species, and Ya and Yb are their monocrop yields.
If LER > 1, intercropping uses land more efficiently than monocropping; LER = 1 means equal efficiency; LER < 1 indicates a disadvantage.


\begin{enumerate}
    \item Discuss monocropping vs intercropping. Mention at least 3 advantages/ 3 disadvantages for each. 
        \begin{table}[h]
            \centering
            \caption{A table with some examples of both advantages and disadvantages of monocropping.}
            \label{tab:table_q_09.1}
            \rowcolors{2}{white}{gray!7}
            \begin{tabular}{p{5cm}|p{5cm}}
            \textbf{Advantages} & \textbf{Disadvantages} \\
            \hline
            1. Simplicity in management and mechanization. & 1. Increased complexity in management. \\
            2. Easier to apply uniform pest and disease control. & 2. Potential for increased competition between crops. \\
            3. Specialization can lead to higher yields of a single crop. & 3. Risk of total crop failure due to pests or diseases. \\
            \end{tabular}
        \end{table}

    2. What is intercropping? \begin{table}[h]
            \centering
            \caption{A table with some examples of both advantages and disadvantages of intercropping.}
            \label{tab:table_q_09.2}
            \rowcolors{2}{white}{gray!7}
            \begin{tabular}{p{5cm}|p{5cm}}
            \textbf{Advantages} & \textbf{Disadvantages} \\
            \hline
            1. Simplicity in management and mechanization. & 1. Increased complexity in management. \\
            2. Easier to apply uniform pest and disease control. & 2. Potential for increased competition between crops. \\
            3. Specialization can lead to higher yields of a single crop. & 3. Risk of total crop failure due to pests or diseases. \\
            \end{tabular}
        \end{table}
    
    \item How would you plan a trial to test the effect of intercropping?  
    \item What is Land Equivalent Ratio?
\end{enumerate}

\newpage
\section{Question 10 - Fertility of Tropical Soils}

\begin{enumerate}
    \item \textbf{Give an overview of the benefits of increasing the content of organic carbon in soil} 
    \item Higher organic carbon improves soil structure, porosity, and water retention, which are vital in semi-arid tropical systems. It enhances nutrient-holding capacity and cation exchange, reducing nutrient losses. It also supports microbial activity, promoting biological soil health. Crops like mauka respond well to soils rich in organic matter ($\ge$3\%), showing better growth and productivity under such conditions.
    \item \textbf{What is the cation exchange capacity of a soil and how does it affect soil fertility?} 
    \item CEC is the soil's ability to hold and exchange positively charged nutrients such as K$^+$, Ca$^{2+}$, and Mg$^{2+}$. Soils with high CEC retain nutrients longer, preventing leaching and ensuring their availability for crops. Organic matter increases CEC, improving fertility and nutrient efficiency, particularly important for tropical soils that are often nutrient-poor.
    \item \textbf{Explain how the content of organic carbon of a soil can be increased.}
    \item Organic carbon can be raised by applying organic fertilizers (manure, compost, humus) and by integrating biomass-producing or nitrogen-fixing crops like achira and ahipa. Llama husbandry and traditional agropastoral practices recycle nutrients through manure. Incorporating crop residues and rotations further builds soil organic matter, enhancing fertility, sustainability, and crop resilience.
\end{enumerate}

\newpage
\section{Question 11 - Legumes as Soil Nutrients Providers}
\textbf{Importance of legumes for the fertility of tropical soils: }

\begin{enumerate}
    \item \textbf{Explain how legumes can play an important role in tropical production systems.} 
    \item Legumes such as Pachyrhizus ahipa play a key role in tropical systems through biological nitrogen fixation, forming symbioses with Rhizobium or Bradyrhizobium that eliminate the need for nitrogen fertilizer. They improve soil fertility by returning nitrogen-rich residues to the soil (up to 215 kg N ha$^{-1}$), lower input costs, and support sustainable, low-input farming. Their adaptability to smallholder systems and capacity to enhance biodiversity make them central to resilient tropical agriculture.
    
    \item \textbf{What is the difference between a legume green manure and a legume cover crop?} 
    \item Both enrich the soil but differ in purpose and management. A green manure is grown primarily to be incorporated into the soil to increase organic matter and nitrogen. A cover crop protects the soil surface, suppresses weeds, and can also provide fodder before decomposition. In the case of Pachyrhizus species, above-ground biomass can serve either role—being incorporated as green manure or used as fodder while still recycling nutrients through manure.
    
    \item \textbf{Are legumes always an advantage for the following crop in the crop rotation?} 
    \item Not always. Although legumes leave a positive nitrogen balance, continuous cultivation can cause pest and nematode buildup. For example, Pachyrhizus erosus performs poorly when grown for more than two consecutive seasons and requires 3-4 years of rotation to prevent pest accumulation. Thus, legumes must be integrated into diverse crop rotations to maintain their benefits.
    
    \item \textbf{Explain how nitrogen fertilizer interacts with the ability of a legume to fix nitrogen.} 
    \item Applying nitrogen fertilizer reduces the need and often the efficiency of symbiotic fixation, as plants preferentially use available mineral N instead of fixing atmospheric N$_2$. In ahipa, inoculation with efficient rhizobia alone maximized yields, showing that external N input is unnecessary. High fixation rates (up to 215 kg N ha$^{-1}$) confirm that these legumes can fully meet their nitrogen needs without fertilizer, lowering both cost and environmental impact.
\end{enumerate}

\newpage
\section{Question 12 - Fertilizers and Manure}
\textbf{Fertilizers in tropical crop production systems:}

\begin{enumerate}
    \item \textbf{What are the advantages and disadvantages of using chemical fertilizers in tropical production systems? }
    \item Advantages:
        \subitem Provide high nutrient concentration, rapidly available to crops.
        \subitem Maximize yield, as seen in mauka where chemical fertilizers produced up to 78.5 t ha$^{-1}$ roots.
        \subitem Allow precise control of nutrient inputs.

    \item Disadvantages:
        \subitem Unsustainable: repeated use degrades soil structure and increases dependency.
        \subitem Do not improve water retention, porosity, or microbial life.
        \subitem Can mask management issues and lead to environmental pollution or economic inefficiency.

    \item \textbf{Compare these with organic fertilizers and explain advantages and disadvantages with these.} 
    \item Advantages:
        \subitem Improve soil structure, porosity, and moisture retention.
        \subitem Enhance cation exchange capacity and biological activity.
        \subitem Recycle nutrients within the farming system; manure adds N, P, and K naturally.

    \item Disadvantages:
        \subitem Lower yields than chemical fertilizers (about 59\% of chemically fertilized yields).
        \subitem Limited availability, often prioritised for main crops like maize or potato.
        \subitem Slow nutrient release, requiring long-term application for noticeable effect.
    
    \item \textbf{Explain the importance of synchrony of supply and demand for nitrogen.} 
    \subitem Nitrogen must be available when the crop needs it most to avoid loss or deficiency. In ahipa, synchrony is achieved through reproductive pruning, which redirects assimilates and nitrogen to roots. Applying manure before planting ensures N availability during early growth. Legumes naturally maintain synchrony through symbiotic fixation, supplying N as needed during development.
    
    \item \textbf{What is a nutrient deficiency symptom in plants and what can be learnt from them?} 
    \subitem Deficiency symptoms, such as poor growth or low yields, indicate imbalances in soil fertility. They reveal which nutrients are limiting productivity. Chemical and nutritional analyses show that deficiencies in N, P, or K reduce root yield, while certain crops like mashua compensate dietary amino acid deficiencies. Observing these symptoms helps identify soil or crop management needs and informs crop selection for balanced nutrition.
\end{enumerate}

\newpage
\section{Question 13 - The Importance of Agrobiodiversity}

\begin{enumerate}
    \item \textbf{Mention four reasons why agro-biodiversity matters for crop breeding.}
        \begin{enumerate}
            \item Provides genetic resources for stress tolerance, such as frost- and drought-resistant landraces of quinoa, cañahua, and native potatoes.
            \item Enhances nutritional and functional diversity, supplying traits like high mineral content and unique compounds (e.g. FOS in yacon, glucosinolates in mashua).
            \item Offers natural pest and disease resistance, seen in mauka, which produces antimicrobial proteins.
            \item Supplies valuable agronomic and quality traits, such as improved starch quality or specific fruit characteristics useful for breeding and industrial purposes.
        \end{enumerate}

    \item \textbf{What are the challenges to work with it? Give at least 5 examples}
        \begin{enumerate}
            \item Economic pressure from short-cycle, marketable crops leading to monoculture.
            \item Loss of traditional knowledge as younger generations migrate away.
            \item Cultural stigma labelling native crops as “poor people's food,” reducing demand.
            \item Agronomic limitations, such as seed shattering in cañahua or high labour needs in ahipa.
            \item Biopiracy and legal barriers under the Nagoya Protocol limiting fair benefit sharing.
            \item Lack of ex situ conservation and research for underutilised species like mauka.
        \end{enumerate}
    
    \item \textbf{How to conserve this agro-biodiversity?}
        \begin{enumerate}
            \item In situ: Maintain diversity on-farm by supporting farmers as biodiversity custodians and monitoring landrace cultivation.
            \item Ex situ: Collect and preserve germplasm in seed banks and research institutions.
            \subitem Additionally, market revalorisation is crucial — promoting the nutritional value of native crops, developing value-added products (e.g. yacon syrup, mashua flour), linking producers with new markets, and training farmers in marketing and processing. These combined actions ensure both conservation and continued use of crop diversity.
        \end{enumerate}
\end{enumerate}

\newpage
\section{Question 14 - Crop Phenotyping}

\begin{enumerate}
    \item \textbf{How can phenotyping approaches support crop production (in future)? }
    \item Phenotyping enables precise measurement of plant traits to identify and improve genotypes best suited to future climate and management conditions. It supports breeding by identifying traits such as drought and frost tolerance, improves management efficiency through understanding growth responses to temperature and density, and allows modelling of thermal time and base temperature to optimise sowing and harvesting. It also enhances nutritional quality and resilience by linking phenotypic traits to productivity and adaptation.
    
    \item \textbf{Can you give some examples for application in the tropics? }
    \begin{itemize}
        \item Cañahua: Germination studies across temperatures (3-24\textdegree C) and sowing depths identify landraces with rapid emergence in cold soils.
        \item Ahipa: Root growth phenotyping correlates yield with heat units and temperature, guiding site selection and planting time.
        \item Capsicum and ARTCs: Chemical phenotyping identifies accessions rich in bioactive compounds (flavonoids, FOS, glucosinolates) for breeding and nutrition.
        \item Participatory sensory phenotyping: Evaluates traits like colour, flavour, and texture to align crop improvement with consumer preference and local traditions.
    \end{itemize}
    
    \item \textbf{Do you see difficulties in its application in the tropics? How to overcome? }
    \item Tropical systems face challenges such as environmental variability, logistical limitations, genetic heterogeneity, and low market value of traditional crops.
        \begin{itemize}
            \item Use multivariate analyses to handle genotype × environment interactions.
            \item Combine quantitative data with farmers' knowledge through participatory methods to overcome data gaps.
            \item Apply standardized morphological descriptors and improve germplasm conservation.
            \item Revalue crops economically by linking phenotyping outcomes to gastronomy and market innovation.
        \end{itemize}
\end{enumerate}

\newpage
\section{Question 15 - Small and Large Scale Farming}

\begin{enumerate}
    \item \textbf{Mention at least two characteristics for each: small and large scale farming systems}
    \item Small-scale farming:
        \begin{itemize}
            \item High agrobiodiversity on limited land, relying on mixed cropping systems like chiru and ananta.
            \item Low input dependence, using traditional tools and manure, guided by local ecological knowledge.
        \end{itemize}

    \item Large-scale farming:
        \begin{itemize}
            \item Monoculture of marketable crops, focusing on yield and commercial export (e.g. quinoa expansion in the Altiplano).
            \item High mechanization and land use, often competing with traditional livestock systems such as llama husbandry.
        \end{itemize}
    
    \item \textbf{Define “sustainable intensification” and explain why some people consider the term self-contradicting.}
    \item Sustainable intensification aims to increase food production while minimizing environmental impact and preserving ecosystem services. It focuses on closing yield gaps through efficient agronomic management rather than increasing input use.
    \item However, some view it as self-contradictory because intensification historically causes biodiversity loss and resource degradation. Combining “sustainability” (biodiversity, low input) with “intensification” (high productivity, specialization) seems paradoxical when intensification itself threatens ecological balance.
    
    \item \textbf{Give 3 examples of sustainable intensification. And explain one of them in detail.}
        \begin{enumerate}
            \item Legume-based systems: Using Pachyrhizus (e.g. ahipa) to fix atmospheric nitrogen, reduce fertilizer need, and enrich soils.
            \item Integrated agropastoral systems: Balancing llama grazing and quinoa cultivation to maintain soil fertility and livelihood resilience.
            \item Improved agronomic management: Optimizing planting density and crop rotation to raise yields sustainably.
        \end{enumerate}
    \item Detailed example:
    \item Tuberous legumes like ahipa form symbioses with Rhizobium and Bradyrhizobium, fixing up to 215 kg N ha$^{-1}$, removing the need for synthetic nitrogen. When residues are left on the field, they restore soil fertility for subsequent crops. This achieves higher productivity with minimal inputs, exemplifying sustainable intensification through natural nutrient cycling and low environmental impact.
\end{enumerate}

\newpage
\section{Question 16 - Fertilizer and Manure in the Tropics}

\begin{enumerate}
    \item \textbf{How to determine how much nutrients need to apply for crop growth? Discuss the practices to reduce nutrient losses and increase nutrient use efficiency.}
        \begin{enumerate}
            \item Nutrient needs in tropical systems are often established through field trials, soil quality targets, or biological autonomy. Empirical testing defines appropriate fertilizer or manure doses (e.g. 60 N and 40 P units, or 7.5 t ha$^{-1}$ manure). Crops like mauka require soils with $\ge$ 3 \% organic matter, while legumes such as ahipa rely on biological nitrogen fixation, eliminating external N requirements.
            
            \item To increase nutrient use efficiency (NUE) and reduce losses:
                \begin{enumerate}
                    \item Promote N fixation via legumes and rhizobial inoculation (fixing 58-215 kg N ha$^{-1}$).
                    \item Apply pruning in root crops to direct nutrients to the economic organ.
                    \item Recycle manure and residues from agropastoral systems every two years to replenish soil nutrients and organic carbon.
                \end{enumerate}
        \end{enumerate}

    \item \textbf{Discuss the advantages and disadvantages of using mineral and organic fertilizers.}
    \begin{table}[h]
        \centering
        \caption{Comparison of mineral and organic fertilizers in tropical systems.}
        \label{tab:table_q_16.2}
        \rowcolors{2}{white}{gray!7}
        \begin{tabular}{p{5cm}|p{5cm}|p{5cm}}
        \textbf{Fertilizer type} & \textbf{Advantages} & \textbf{Disadvantages} \\
        \hline
        Mineral Fertilizers (Chemical) & 
        \begin{itemize}
            \item High, immediate yield response (e.g. mauka 78.5 t ha$^{-1}$ roots).
            \item Concentrated nutrients easy to apply and control.
            \item Precise nutrient management for specific crop needs.
        \end{itemize} & 
        \begin{itemize}
            \item Does not improve soil structure or microbial health.
            \item Enhance cation exchange capacity and microbial activity.
            \item Recycle nutrients within the farming system.
        \end{itemize} \\
        Organic Fertilizers (Manure) & 
        \begin{itemize}
            \item Enhances physical (structure, porosity), chemical (CEC), and biological (microbial) properties.
            \item Recycles farm nutrients and supports sustainability.
            \item Provides N, P, and K naturally.
        \end{itemize} & 
        \begin{itemize}
            \item Produces lower yields ($\approx$ 59 \% of chemical fertilizer).
            \item Limited availability and labour-intensive handling.
            \item Often reserved for main crops like potato and maize.
            \item Slow nutrient release, requiring long-term application.
        \end{itemize} \\
        \end{tabular}
    \end{table}
\end{enumerate}

Summary:
Chemical fertilizers maximise short-term yield, while organic fertilizers sustain long-term soil fertility and resilience. Combining both judiciously allows higher efficiency and ecological stability in tropical production systems.

\newpage
\section{Question 17 - Fertility of Tropical Soils}

\begin{enumerate}
    \item \textbf{What is soil fertility? How soil organic carbon helps to improve soil fertility?} 
        \begin{enumerate}
            \item Soil fertility is the soil's capacity to provide essential nutrients and physical conditions for healthy crop growth. In tropical and high-altitude regions, fertility determines crop establishment and yield potential under challenging conditions.
            \item Soil organic carbon (SOC), as part of organic matter, enhances fertility through:
                \begin{itemize}
                    \item Physical improvement: increases soil structure, porosity, and water retention, essential in semi-arid climates.
                    \item Chemical enrichment: boosts nutrient retention and cation exchange capacity.  
                    \item Biological enhancement: supports microbial activity that aids nutrient cycling.
                \end{itemize}
            \item Crops like mauka perform best in soils with $\ge$ 3 \% organic matter, showing SOC's key role in productivity.
        \end{enumerate}
    
    \item \textbf{What is the cation exchange capacity (CEC) of a soil? and how does CEC affect soil fertility?} 
        \item CEC is the soil's ability to hold and exchange positively charged nutrients (e.g. K$^+$, Ca$^{2+}$, Mg$^{2+}$) on clay and organic matter surfaces.
        \item A high CEC means:
            \begin{itemize}
                \item Better nutrient retention and reduced leaching.
                \item Greater availability of macronutrients like phosphorus and potassium.
                \item Enhanced chemical fertility, maintaining continuous nutrient supply for plants.
            \end{itemize}
        \item Thus, soils rich in organic matter and clay—those with high CEC—are more fertile, resilient, and productive under tropical conditions.
\end{enumerate}

\newpage
\section{Question 18 - Legumes as Soil Nutrients Providers}

\begin{enumerate}
    \item \textbf{What is the difference between a legume green manure and a legume cover crop? Discuss with examples, advantages and disadvantages of each method.}
        \item Green manure legumes are grown to enrich the soil by incorporating their biomass directly into it.
            \begin{itemize}
                \item Example: Leaving Pachyrhizus (yam bean) tops in the field after harvest.
                \item Advantages: Returns large amounts of fixed nitrogen, improving soil fertility and reducing fertilizer needs.
                \item Disadvantages: Biomass cannot be used as fodder, reducing short-term income.
            \end{itemize}
        \item  Cover crops protect soil from erosion and provide feed or mulch while indirectly adding nitrogen through residues or manure.
            \begin{itemize}
                \item Example: Dried P. erosus hay used as animal fodder mixed with maize or lucerne.
                \item Advantages: Supplies protein-rich feed and recycles nutrients through manure.
                \item Disadvantages: Risk of toxicity from rotenone if not properly managed; requires pruning and monitoring.
            \end{itemize}

    \item \textbf{What are the advantages and disadvantages of legumes in cropping system?}
        \item Advantages:
            \begin{itemize}
                \item Fix atmospheric nitrogen (up to 215 kg N ha$^{-1}$), reducing fertilizer costs.
                \item Improve soil fertility and sustainability through rotation or intercropping.
                \item Provide diverse outputs: protein-rich seeds, roots, and fodder.
            \end{itemize}
        \item Disadvantages:
            \begin{itemize}
                \item Labour-intensive management (e.g. reproductive pruning).
                \item Toxic compounds (rotenone) in unpruned plants limit use.
                \item Pest and nematode buildup requires long rotations (3-4 years).
            \end{itemize}

    \item \textbf{What are potential constraints in adaptation of legumes by small holder farmers?}
            \begin{itemize}
                \item Labour demands: Manual pruning and management are time-consuming.
                \item Market limitations: Weak demand and poor transport hinder profitability.
                \item Competition: Cash crops replace legumes in limited land systems.
                \item Knowledge and breeding gaps: Limited research and improved landraces.
                \item Agronomic complexity: Perennial habit complicates short-cycle production.
            \end{itemize}
\end{enumerate}

\newpage
\section{Question 19 - Tropical Crop Physiology}

\begin{enumerate}
    \item \textbf{What are the four major environmental factors influencing crop evapotranspiration (ET)? How do they affecting ET?} 
        \begin{enumerate}
            \item ET combines evaporation from soil and transpiration from plants, driven by the energy and water exchange between the surface and atmosphere.
            \item The four key environmental factors are:
                \begin{itemize}
                    \item Temperature: Governs atmospheric demand for water vapour through the vapour pressure deficit. High temperatures increase ET, while cold conditions reduce it and slow plant growth.
                    \item Water availability: ET depends on soil moisture; when rainfall is less than crop water requirement, transpiration and growth decline. Rapid germination helps crops exploit available soil water before dry periods.
                    \item Solar radiation: Provides the latent heat for water evaporation and drives physiological processes such as flowering and photosynthesis, influencing ET rates.
                    \item Air movement (wind): Removes humid air around leaves, maintaining the vapour gradient and enhancing ET. In calm conditions, ET is lower due to limited air exchange.
                \end{itemize}
        \end{enumerate}

    \item \textbf{Why is stomatal conductance important in controlling crop transpiration? How is it regulated under drought stress? }
        \begin{enumerate}
            \item Stomatal conductance determines how open stomata are, controlling transpiration and CO$_2$ uptake. By adjusting stomatal aperture, crops regulate water loss while maintaining photosynthesis. Efficient water users like ahipa exhibit high water-use efficiency due to tight stomatal control.
            \item Under drought stress, regulation occurs through several adaptive responses:
                \begin{itemize}
                    \item Reduced leaf area: Some crops, like cassava, shed foliage to lower transpiring surface.
                    \item Early water use: Species such as cañahua germinate rapidly to use moisture before drought onset.
                    \item Intrinsic drought tolerance: Crops like mauka and cañahua maintain low transpiration or resilient stomatal function, allowing survival under low soil water conditions.
                \end{itemize}
            \item Together, these mechanisms balance water conservation and productivity in tropical, high-altitude environments.
        \end{enumerate}
\end{enumerate}


\section{Question 20 - Rice}

\begin{enumerate}
    \item \textbf{What is the difference between upland and lowland rice production systems?}
        \begin{itemize}
            \item Lowland (paddy) rice is cultivated in flooded or semi-flooded fields where water levels are actively managed using bunds or ridges. The soil remains saturated during most of the crop cycle, which suppresses weeds and supports anaerobic microbial activity. This system relies on controlled irrigation or heavy rainfall.
            \item Upland rice grows under rainfed, non-flooded conditions in dryland fields. It depends entirely on rainfall, making it more vulnerable to drought and requiring well-drained soils.
        \end{itemize}

    \item \textbf{Provide examples of countries where each of these production systems can be found.}
        \begin{itemize}
            \item Lowland (paddy) rice: Found in China (Sichuan and Chengdu provinces), where crops like yam bean or soybean are grown on the ridges between rice paddies.
            \item Upland rice: Common in Central American dryland systems where crops are cultivated under rainfed conditions similar to those described for yam bean fields.
        \end{itemize}
\end{enumerate}


\section{Question 21 - Legumes as Nutrient Providers}

\begin{enumerate}
    \item \textbf{How can legumes play an important role in tropical production systems? Detail the mechanism. }
        \begin{itemize}
            \item Legumes improve tropical soil fertility through Biological Nitrogen Fixation (BNF). Symbiotic Rhizobium or Bradyrhizobium bacteria in root nodules convert atmospheric N$_2$ into ammonia, which the plant uses for growth. This process supplies the entire N requirement of crops like Pachyrhizus ahipa without external fertilizers. When residues are left in the field, a substantial portion of the fixed N (up to 215 kg N ha$^{-1}$) enriches the soil, benefiting subsequent crops. Legumes also reduce input costs, enhance sustainability, and can be intercropped for diversified yields and resilience.
        \end{itemize}

    \item \textbf{Are legumes always an advantage for the following crop in the crop rotation? How to assess if a legume is advantageous for the following crop?}
        \begin{enumerate}
            \item Legumes are generally beneficial due to their N contribution but may not always be advantageous if pests or nematodes accumulate. For example, Pachyrhizus erosus performs poorly after continuous cultivation and requires 3-4 years before replanting.
            \item To assess advantage:
                \begin{itemize}
                    \item Measure soil N balance or fixed N amount.
                    \item Observe yield response of the following crop.
                    \item Evaluate nodule efficiency (e.g. by rhizobial inoculation). A clear yield or fertility improvement confirms advantage.
                \end{itemize}
        \end{enumerate}

    \item \textbf{How do nitrogen fertilizers interact with the ability of a legume to fix nitrogen?}
        \begin{itemize}
            \item External nitrogen fertilization makes BNF redundant or suppressed, as the legume's symbiosis downregulates when mineral N is available. Efficient fixers like P. ahipa meet their own N needs biologically, so added fertilizer is unnecessary and economically wasteful.
        \end{itemize}
\end{enumerate}

\newpage
\section{Question 22 - Minor Cereals}

\begin{enumerate}
    \item \textbf{Give examples of minor cereals with local importance in the tropics}
        \begin{itemize}
            \item In tropical highlands, locally important minor cereals include cañahua (Chenopodium pallidicaule) and quinoa (Chenopodium quinoa), traditional Andean grains cultivated between 3800-4200 m a.s.l. for their high protein and nutrient value. Among true cereals (Poaceae), maize (Zea mays) and wheat (Triticum aestivum) are also regionally significant as short-cycle, market-oriented crops that increasingly replace traditional polycultures.
        \end{itemize}

    \item \textbf{How is the yield of sorghum and millet in the tropics compared to other cereals?}
        \begin{itemize}
            \item Although direct yield data are not given, minor cereals like cañahua and quinoa generally produce lower yields than major cereals under optimal conditions but perform more reliably in extreme environments. Their capacity to germinate and mature quickly under drought, cold, or heat stress makes them superior for food security in marginal tropical and high-altitude areas.
        \end{itemize}

    \item \textbf{As a consequence, how are they commonly grown in the tropics (what type of cropping system)?}
        \begin{itemize}
            \item Minor cereals are usually grown in low-input, diversified systems, such as:
                \begin{itemize}
                    \item Agropastoral systems, where quinoa and other grains are integrated with llama husbandry.
                    \item Rotations with potatoes or legumes, taking advantage of residual fertility.
                \end{itemize}
            \item However, in commercial areas, these crops are increasingly cultivated as monocultures, replacing mixed traditional fields and creating land-use competition.
        \end{itemize}
    
    \item \textbf{What are the 2 main factors determining the choice of cropping system?}
        \begin{enumerate}
            \item Environmental adaptation and resilience - selection of crops suited to cold, drought, and poor soils.
            \item Socio-economic viability and labour demand - preference for short-cycle, less labour-intensive, and marketable crops over traditional, labour-heavy systems.
        \end{enumerate}
\end{enumerate}
