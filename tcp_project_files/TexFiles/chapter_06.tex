\chapter{Exam Questions and Answers}
\setlength{\headheight}{12.71342pt}
\addtolength{\topmargin}{-0.71342pt}

This chapter of the course notes compiles the exam questions for the course held in November 2025, along with their respective answers prepared by me. The purpose of this section is twofold: firstly, to provide a reflective exercise that consolidates understanding of the course material; and secondly, to document my comprehension of the course topics as assessed through the exam questions.

\vspace{1em}
To ensure citation accuracy and academic transparency, NotebookLM has been employed as the primary generative AI platform. Its use has focused on verifying that all citations accurately reference the uploaded course materials and lecture slides provided by the professors. Beyond citation control, this section also represents an ongoing exploration of prompt engineering - refining interaction design to optimise AI output quality, precision, and academic reliability. Through this approach, the work aims to maintain a high academic standard while enhancing clarity, structure, and depth in written responses.

\vspace{1em}
There are a total of 17 questions in the exam, each comprising between three and five sub-questions. The numbering of the sections in this chapter corresponds directly to the numbering of the exam questions, ensuring a clear and consistent structure throughout. Questions 1-9 address aspects related to crop physiology, while questions 10-17 focus on fruit quality, maturity, and usability. Each question is presented below, followed by its respective sub-questions and answers.

\newpage
\section{Question 1 - Highland Crops}
\textbf{How would you make the association between crop calendar and climate change. Specify links between crop phenology and climate knowledge. You can use an example:}

\section{Question 2 - Highland Crops}
\textbf{Which are the social constrains when introducing new varieties? And other factors of agronomic importance? }

\section{Question 3 - Intercropping}
\textbf{Explain the concept of intercropping and provide potential benefits of practicing intercropping in agriculture.}

\section{Question 4 - Intercropping}
\textbf{Discuss the challenges and potential disadvantages of intercropping in modern agricultural systems. Provide examples of situations where intercropping may not be the best strategy.}

\begin{figure}[h]
    \centering
    \includegraphics[width=0.8\textwidth]{Figures/tcp_q_05.JPG}
    \caption{An example of intercropping with maize and beans, demonstrating the complementary growth patterns and resource use of the two crops.}
\end{figure}



