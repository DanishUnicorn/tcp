\chapter{Exam Questions and Answers}
\setlength{\headheight}{12.71342pt}
\addtolength{\topmargin}{-0.71342pt}

This chapter of the course notes compiles the exam questions for the course held in November 2025, along with their respective answers prepared by me. The purpose of this section is twofold: firstly, to provide a reflective exercise that consolidates understanding of the course material; and secondly, to document my comprehension of the course topics as assessed through the exam questions.

\vspace{1em}
To ensure citation accuracy and academic transparency, NotebookLM has been employed as the primary generative AI platform. Its use has focused on verifying that all citations accurately reference the uploaded course materials and lecture slides provided by the professors. Beyond citation control, this section also represents an ongoing exploration of prompt engineering - refining interaction design to optimise AI output quality, precision, and academic reliability. Through this approach, the work aims to maintain a high academic standard while enhancing clarity, structure, and depth in written responses.

\vspace{1em}
There are a total of 17 questions in the exam, each comprising between three and five sub-questions. The numbering of the sections in this chapter corresponds directly to the numbering of the exam questions, ensuring a clear and consistent structure throughout. Questions 1-9 address aspects related to crop physiology, while questions 10-17 focus on fruit quality, maturity, and usability. Each question is presented below, followed by its respective sub-questions and answers.

\newpage
\section{Question 1 - Highland Crops}
\textbf{How would you make the association between crop calendar and climate change. Specify links between crop phenology and climate knowledge. You can use an example:}


\section{Question 2 - Highland Crops}
\textbf{Which are the social constrains when introducing new varieties? And other factors of agronomic importance? }


\section{Question 3 - Intercropping}
\textbf{Explain the concept of intercropping and provide potential benefits of practicing intercropping in agriculture.}


\section{Question 4 - Intercropping}
\textbf{Discuss the challenges and potential disadvantages of intercropping in modern agricultural systems. Provide examples of situations where intercropping may not be the best strategy.}


\section{Question 5 - Seed Germination}
\textbf{Germination test of three seed lots of cowpea supplied by a farmer }

\begin{figure}[h]
    \centering
    \includegraphics[width=0.8\textwidth]{Figures/tcp_q_05.JPG}
    \caption{The relation between percent germinated seed over time of three cowpea seed lots.}
    \label{fig:tcp_q_05}
\end{figure}

\begin{enumerate}
    \item \textbf{Describe the curves.}
        \begin{enumerate}
            \item this curve
            \item this curve
            \item this curve
        \end{enumerate}
    \item \textbf{The germination curves represent developments as expected for an ordinary germination test for the three seed lots?} 
    \item \textbf{How can you estimate the germination curve based on count over time using a mathematical model.}
    \item \textbf{Which parameters are necessary in the mathematical model?}
    \item \textbf{How can you test whether the curves for seed lots a and b are statistically significantly different.}
    \item \textbf{What could be the reason why the curve for seed lot c has another shape than seed lots a and b?}
    \item \textbf{If seed lots a and b are actually identical, but curve a is a result of an ordinary germination test, what kind of test could then result in curve b.}
\end{enumerate}


\section{Question 6 - Sugar Production}
\begin{enumerate}
    \item What is sugar?  
    \item Mention all the crops you known which are used for sugar production.  
    \item Sugar cane can be used for several purposes. Mention these.  
    \item Explain how sugar cane normally is established in the field. 
    \item Is it necessary to establish a new sugar cane crop each year?  
    \item Why and when do farmers many places in the world ignite sugar cane fields? 
    \item After sugar canes have been processed in a factory, which types of waste products are produced and what can they be used for? 
\end{enumerate}


\section{Question 7 - Cropping Systems}

A small subsistence farm is placed in a semi-arid area in the highlands of Guatemala (500 mm of annual precipitation). Crops are grown in rotation with a short fallow period, and fertilizer is not used on the subsistence crops. The field is not irrigated. Answer the following questions for a field grown with maize intercropped with bean. 

\vspace{0.5em}
Indicate the approximate onset and duration of the rainy season, give sowing and harvesting time for the two crops, and estimate a realistic yield for maize monocrop. Do you think the yields can improve with the intercropping, why?. 

\vspace{0.5em}
Suggest crop establishment for the two crops. What could be the advantages of intercropping the two species? 


\section{Question 8 - Cropping Systems}
\textbf{Quinoa in Bolivia:}

\vspace{0.5em}
A farmer close to Titicaca Lake in Bolivia grows quinoa as one of his main crops. His village receives about 800 mm rain per year. 

\vspace{0.5em}
How is rainfall and temperature distributed over the year? 

\vspace{0.5em}
When is quinoa sown and harvested? 

\vspace{0.5em}
Which other crops may the farmer grow? 


\section{Question 9 - Intercropping}

\begin{enumerate}
    \item Discuss monocropping vs intercropping. Mention at least 3 advantages/ 3 disadvantages for each. 
        \begin{table}[h]
            \centering
            \caption{A table with some examples of both advantages and disadvantages of monocropping.}
            \label{tab:table_q_09.1}
            \rowcolors{2}{white}{gray!7}
            \begin{tabular}{p{5cm}|p{5cm}}
            \textbf{Advantages} & \textbf{Disadvantages} \\
            \hline
            1. Simplicity in management and mechanization. & 1. Increased complexity in management. \\
            2. Easier to apply uniform pest and disease control. & 2. Potential for increased competition between crops. \\
            3. Specialization can lead to higher yields of a single crop. & 3. Risk of total crop failure due to pests or diseases. \\
            \end{tabular}
        \end{table}

    2. What is intercropping? \begin{table}[h]
            \centering
            \caption{A table with some examples of both advantages and disadvantages of intercropping.}
            \label{tab:table_q_09.2}
            \rowcolors{2}{white}{gray!7}
            \begin{tabular}{p{5cm}|p{5cm}}
            \textbf{Advantages} & \textbf{Disadvantages} \\
            \hline
            1. Simplicity in management and mechanization. & 1. Increased complexity in management. \\
            2. Easier to apply uniform pest and disease control. & 2. Potential for increased competition between crops. \\
            3. Specialization can lead to higher yields of a single crop. & 3. Risk of total crop failure due to pests or diseases. \\
            \end{tabular}
        \end{table}
    
    \item How would you plan a trial to test the effect of intercropping?  
    \item What is Land Equivalent Ratio?
\end{enumerate}


\section{Question 10 - Fertility of Tropical Soils}

\begin{enumerate}
    \item Give an overview of the benefits of increasing the content of organic carbon in soil 
    \item What is the cation exchange capacity of a soil and how does it affect soil fertility 
    \item Explain how the content of organic carbon of a soil can be increased. 
\end{enumerate}
