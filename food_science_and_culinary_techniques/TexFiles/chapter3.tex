\chapter{Lecture Notes}
\setlength{\headheight}{12.71342pt}
\addtolength{\topmargin}{-0.71342pt}

\section*{1\texorpdfstring{\textsuperscript{st}}{st} Lecture - Plants and Food Colours}
\subsection*{Lecture Goals}
After this lecture, the students will be able to:
\begin{highlight}
    \begin{itemize}
    \item Describe the structures of fruits and vegetables
    
    \item Identify structural carbohydrates in plants \& changes they undergo during ripening and processing
    
    \item Describe important factors responsible for texture, colours, flavours, and taste of plants
    \end{itemize}
\end{highlight}

\subsection*{Plant Organs}
Plants have different organs, each serving a specific function. Table \ref{tab:L01_plant_organs} shows some plant organs and examples of fruits and vegetables which has the following trait.
\begin{table}[h]
    \centering
    \caption{A table showing plant organs and examples}
    \label{tab:L01_plant_organs}
    \begin{tabular}{c|c}
    \textbf{Plant Organs} & \textbf{Examples} \\
    \hline
    Roots & Carrots, radishes \\

    Stems, stalks, tubers, rhizomes & Potatoes, ginger \\

    Leaves & Spinach, lettuce \\

    Flowers & Cauliflower, broccoli \\

    Fruits & Apples, oranges \\

    Seeds & Peas, beans \\
    \end{tabular}
\end{table}



