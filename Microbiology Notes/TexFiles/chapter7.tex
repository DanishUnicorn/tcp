\setlength{\headheight}{12.71342pt}
\addtolength{\topmargin}{-0.71342pt}

\chapter{Abbreviations and Explanations}

\begin{longtable}{| p{5cm} | p{2cm} | p{7.5cm} |}
    \hline
    \textbf{Topic} & \textbf{Abb.} & \textbf{Description} \\ 
    \hline
    \endfirsthead
    
    \hline
    \textbf{Topic} & \textbf{Abb.} & \textbf{Description} \\ 
    \hline
    \endhead
    
    \hline \multicolumn{3}{r}{\textit{Continued on next page}} \\ 
    \endfoot
    
    \hline
    \endlastfoot
    
    \textbf{Acetic Acid Bacteria} & \textbf{AAB} & \textit{Gram-negative, oxidase-negative, catalase-positive, ellipsoidal to rod-shaped bacteria that grow mesophilically and oxidize sugars and ethanol into acetic acid.} \\
    \hline
    \textbf{Aerobic Endospore-Forming Bacteria} & \textbf{AEB} & \textit{Bacteria capable of producing endospores in the presence of oxygen.} \\
    \hline
    \textbf{Acetic Acid-Ethanol Substrate} & \textbf{AE} & \textit{A mixture of acetic acid and ethanol used by certain bacteria in fermentation processes.} \\
    \hline
    \textbf{Aerobic} &  & \textit{Organisms that require oxygen for growth.} \\
    \hline
    \textbf{Agar Diffusion Assays} &  & \textit{A method used to evaluate the antimicrobial activity of substances by measuring the zone of inhibition on an agar plate.} \\
    \hline
    \textbf{Alkaline} &  & \textit{A substance with a pH greater than 7, often used in the context of bacterial growth conditions.} \\
    \hline
    \textbf{Amylase} &  & \textit{An enzyme that catalyzes the hydrolysis of starch into sugars.} \\
    \hline
    \textbf{Average Nucleotide Identity} & \textbf{ANI} & \textit{A genomic similarity measure between two organisms based on their nucleotide sequences.} \\
    \hline
    \textbf{API 20C Kit} &  & \textit{A biochemical kit used to identify yeasts by assessing various fermentation and assimilation tests.} \\
    \hline
    \textbf{Bacteriocins} &  & \textit{Proteinaceous toxins produced by bacteria to inhibit the growth of similar or closely related bacterial strains.} \\
    \hline
    \textbf{Biofilm Formation} &  & \textit{The process by which microbial cells adhere to surfaces and form protective colonies.} \\
    \hline
    \textbf{Catalase-positive} &  & \textit{Bacteria that produce the enzyme catalase, which breaks down hydrogen peroxide into water and oxygen.} \\
    \hline
    \textbf{Catalase-negative} &  & \textit{Bacteria that do not produce the enzyme catalase.} \\
    \hline
    \textbf{Cell Lysis} &  & \textit{The breakdown or destruction of a cell.} \\
    \hline
    \textbf{Class I Bacteriocin} &  & \textit{Bacteriocins characterized by containing unusual amino acids such as lanthionine.} \\
    \hline
    \textbf{Class II Bacteriocin} &  & \textit{Small, heat-stable peptides without unusual amino acids, often effective against Gram-positive bacteria.} \\
    \hline
    \textbf{Class IIa Bacteriocin} &  & \textit{Pediocin-like bacteriocins, targeting \textit{Listeria} species.} \\
    \hline
    \textbf{Class IIb Bacteriocin} &  & \textit{Two-component bacteriocins requiring two different peptides for their activity.} \\
    \hline
    \textbf{Class III Bacteriocin} &  & \textit{Large, heat-labile bacteriocins.} \\
    \hline
    \textbf{Class IV Bacteriocin} &  & \textit{Bacteriocins that require a lipid or carbohydrate moiety for activity.} \\
    \hline
    \textbf{Dendritic Cells} &  & \textit{Immune cells that process antigens and present them to other immune cells.} \\
    \hline
    \textbf{Endospore} &  & \textit{A tough, dormant form of bacteria that can survive harsh conditions.} \\
    \hline
    \textbf{Ex vivo} &  & \textit{Studies or processes performed outside the living organism but in an environment similar to the body.} \\
    \hline
    \textbf{Extragenic} &  & \textit{Pertaining to DNA regions located outside the coding genes.} \\
    \hline
    \textbf{Genome} &  & \textit{The complete set of genes or genetic material in an organism.} \\
    \hline
    \textbf{Gastrointestinal} & \textbf{GI} & \textit{Pertaining to the stomach and intestines.} \\
    \hline
    \textbf{Gram-negative} &  & \textit{Bacteria that do not retain the crystal violet stain in Gram's method of staining.} \\
    \hline
    \textbf{Gram-positive} &  & \textit{Bacteria that retain the crystal violet stain in Gram's method of staining.} \\
    \hline
    \textbf{High-Performance Liquid Chromatography} & \textbf{HPLC} & \textit{A technique in analytical chemistry used to separate, identify, and quantify components in a mixture.} \\
    \hline
    \textbf{Illumina} &  & \textit{A next-generation sequencing technology commonly used for DNA sequencing.} \\
    \hline
    \textbf{In vitro} &  & \textit{Studies or processes performed outside a living organism, usually in a test tube or culture dish.} \\
    \hline
    \textbf{In vivo} &  & \textit{Studies or processes performed inside a living organism.} \\
    \hline
    \textbf{Internal Transcribed Spacer Region} & \textbf{ITS Region} & \textit{A region in the genetic code often used in sequencing to identify fungal species.} \\
    \hline
    \textbf{Lactic Acid Bacteria} & \textbf{LAB} & \textit{Gram-positive bacteria that produce lactic acid as a byproduct of carbohydrate fermentation.} \\
    \hline
    \textbf{Lantibiotic} &  & \textit{A class of bacteriocins that contain unusual amino acids like lanthionine.} \\
    \hline
    \textbf{Lipase} &  & \textit{An enzyme that breaks down lipids (fats).} \\
    \hline
    \textbf{Matrix-Assisted Laser Desorption/Ionization Time of Flight Mass Spectrometry} & \textbf{MALDI-TOF MS} & \textit{A technique used for identifying microorganisms by analyzing their protein patterns.} \\
    \hline
    \textbf{Metabolite} &  & \textit{A substance formed in or necessary for metabolism.} \\
    \hline
    \textbf{Multilocus Sequence Analysis} & \textbf{MLSA} & \textit{A method for characterizing species using sequences of multiple genes.} \\
    \hline
    \textbf{Multilocus Sequence Typing} & \textbf{MLST} & \textit{A technique for characterizing bacterial species by comparing the sequences of housekeeping genes.} \\
    \hline
    \textbf{Nanopore Amplicon Sequencing} &  & \textit{A type of sequencing technology that reads long DNA or RNA fragments.} \\
    \hline
    \textbf{Nisin} &  & \textit{A lantibiotic bacteriocin used as a food preservative.} \\
    \hline
    \textbf{Operon} &  & \textit{A cluster of genes under the control of a single promoter, often found in prokaryotes.} \\
    \hline
    \textbf{PCR-Restriction Fragment Length Polymorphism} & \textbf{PCR-RFLP} & \textit{A technique used for identifying differences in homologous DNA sequences.} \\
    \hline
    \textbf{Peptides} &  & \textit{Short chains of amino acids.} \\
    \hline
    \textbf{Pediocin} &  & \textit{A Class IIa bacteriocin effective against \textit{Listeria} species.} \\
    \hline
    \textbf{Phenotypic} &  & \textit{Pertaining to the observable characteristics of an organism.} \\
    \hline
    \textbf{Pulsed-Field Gel Electrophoresis} & \textbf{PFGE} & \textit{A technique used for separating large DNA molecules by applying a varying electric field.} \\
    \hline
    \textbf{Phagocytosis} &  & \textit{The process by which cells engulf particles or microorganisms.} \\
    \hline
    \textbf{Prebiotic} &  & \textit{A substance that stimulates the growth or activity of beneficial microorganisms in the gut.} \\
    \hline
    \textbf{Probiotic} &  & \textit{Live microorganisms that, when administered in adequate amounts, confer a health benefit to the host.} \\
    \hline
    \textbf{Protease} &  & \textit{An enzyme that breaks down proteins and peptides.} \\
    \hline
    \textbf{Proteolytic} &  & \textit{Referring to the breakdown of proteins or peptides.} \\
    \hline
    \textbf{Quorum Sensing} & \textbf{QS} & \textit{A process of cell-to-cell communication in bacteria to coordinate group behaviors based on population density.} \\
    \hline
    \textbf{Random Amplified Polymorphic DNA} & \textbf{RAPD} & \textit{A technique used for typing microorganisms by amplifying random DNA sequences.} \\
    \hline
    \textbf{Repetitive Extragenic Palindromic PCR} & \textbf{Rep-PCR} & \textit{A PCR method used to amplify repetitive DNA elements in bacterial genomes for differentiation purposes.} \\
    \hline
    \textbf{Short-Chain Fatty Acids} & \textbf{SCFA} & \textit{Fatty acids with fewer than six carbon atoms, produced by microbial fermentation in the gut.} \\
    \hline
    \textbf{Whole Genome Sequencing} & \textbf{WG-sequencing} & \textit{A method for sequencing the entire genome of an organism.} \\
    \end{longtable}




